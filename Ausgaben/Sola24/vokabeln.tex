\section*{Finnisch für Pfadis}

Damit ihr hier in Finnland mit vielen Leuten in Kontakt kommen könnt, haben wir für euch die wichtigsten Wörter zusammengestellt:

\begin{itemize}
	\item Keine Unterscheidung von „er“, „sie“ oder „es“ – einfach nur „hän“
	\item Keine Artikel wie „der“, „die“ oder „das“
	\item Sage und schreibe 15 (!) Fälle (das Deutsche hat nur 4)
	\item Das Verb „haben“ gibt es nicht
\end{itemize}

\begin{table}[h!]
	\centering
	\small
	\begin{tabular}{m{6cm}m{6cm}}
	\hline
	Ja & Joo / kyllä \\ \hline
	Nein & ei \\ \hline
	Wir sind Pfadfinder aus Deutschland & Olemme partiolaisia/ Scouts Saksasta \\ \hline
	Bitte / Danke & Ole hyvä / kiitos \\ \hline
	Entschuldigung & anteeksi \\ \hline
	Ich verstehe dich nicht. & En ymmärrä sinua \\ \hline
	Ich spreche kein finnisch & En puhu suomea \\ \hline
	Sprechen Sie Englisch ? & Puhutko englantia? \\ \hline
	Wo ist das WC? & Missä on vessa? \\ \hline
	Hallo & Hei/ moi \\ \hline
	Tschüss & Näkemiin/ hei hei/ moi moi \\ \hline
	Dürfen wir bei Ihnen zelten? & Voimmeko leiriytyä kanssanne? \\ \hline
	Bär & Karhu \\ \hline
	See & Järvi \\ \hline
	Rentier & poro \\ \hline
	Elch & hirvi \\ \hline
	Wald & metsä \\ \hline
	Sommerhaus am See & mökki \\ \hline
	Sisu ist, wenn du nicht aufgibst oder wenn du scheiterst und es trotzdem weiter versuchst & sisu \\ \hline
	Besserwisser*in & besservisseri \\ \hline
	Ich heiße… & Nimeni on... \\ \hline
	Wie heißt du? & Mikä sinun nimesi on? \\ \hline
	Ich heiße… & Minun nimeni on ... \\ \hline
	Wie geht es dir? & Mitä kuuluu? \\ \hline
	Mir geht es gut / schlecht & Minulla menee hyvin/ Minusta tuntuu pahalta \\ \hline
	Hilfe & Apua \\ \hline
	Ich möchte das & Haluan tämän \\ \hline
	Freitag / Samstag / Sonntag & Perjantai/ Lauantai/ sunnuntai \\ \hline
	Geöffnet / geschlossen & avoin / suljettu \\ \hline
	\small vegan, Eier, Milch, Butter, Käse, Honig & \footnotesize vegaani, kananmuna, maito, voi, juusto, hunaja \\ \hline
\end{tabular}
\end{table}
	