\section*{Fun Facts}

\begin{itemize}
	\item 41 Nationalparks
	\item 9. April ist Mikael-Agricola-Tag und Tag der finnischen Sprache
	\item „Die zwei größten Sünden sind das Furzen in der Sauna sowie das Zufußgehen auf der Loipe.“
	\item 188.000 Seen, 80.000 Elche, bis zu 200.000 Rentiere und mehr als 1.500 Braunbären
	\item 5,5 Millionen Einwohner, 18 Personen pro Quadratkilometer
	\item Das Land ist auch das glücklichste Land der Welt laut World Happiness Report
	\item Etwa 75 Prozent seiner Oberfläche sind mit Wäldern bedeckt. Es ist zudem die Heimat des größten Schärenmeers der Welt, umfasst Europas größtes Seengebiet und die letzte ungezähmte Wildnis – Lappland.
	\item Finnlands Hauptstadt Helsinki ist für Design und Architektur bekannt.
	\item Finnland ist auch ein sicheres Reiseland: 11 von 12 verlorenen Geldbörsen werden an ihre Besitzer zurückgegeben.
	\item Vor über 10.000 Jahren ließen sich die ersten bekannten Ureinwohner in Finnland nieder.
	\item Viele Jahrhunderte später wurde das Gebiet, das das heutige Finnland umfasst, von den Vorgängern der Schweden und Russen erobert.
	\item 1809 wurde Finnland ein autonomer Teil des Russischen Reiches, erlangte jedoch 1917 die volle Unabhängigkeit.
	\item Finnland war auch das erste europäische Land, das Frauen im Jahr 1906 das Wahlrecht verlieh.
	\item Während des Zweiten Weltkriegs behielt Finnland seine Unabhängigkeit und nimmt seither eine neutrale Haltung in der Geopolitik ein.
	\item Heute ist Finnland Teil der Europäischen Union.
	\item In Finnland gibt es jährlich die Air Guitar WM, Sumpffußball oder auch Gummistiefel-Weitwurf, Heavy-Metal-Knitting, Handyweitwurf, Wettkampf im Sauna anheizen.
	\item Die Finnen haben weltweit den höchsten pro-Kopf-Verbrauch an Kaffee und europaweit die meiste verzehrte Menge Eis pro Jahr und Kopf.
	\item Es gibt mehr Saunen als Autos.
	\item Donald Duck heißt in Finnland Aku Ankka.
	\item Es gibt eine Rentier-Warn-App.
	\item Erfindungen aus Finnland: Schlittschuhe, der Molotowcocktail, das Computerspiel „Angry Birds“, „Erwise“ (der erste Internet-Browser mit einer Benutzeroberfläche), der Herzfrequenz-Monitor, salziges Lakritz („Salmiakki“), das „Linux“-Betriebssystem, die SMS.
\end{itemize}