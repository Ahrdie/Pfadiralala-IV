\section*{Wag es!}
	\footnotesize
	\centering
	\begin{tabular}{m{2cm}m{10cm}}
	\includegraphics[width=2cm]{Ausgaben/Sola24/Grafiken/heart.png} & \textbf{Wag es, das Leben zu lieben!} \newline
	Du bist selbstverantwortlich für dein Leben. Für dein Leben schreibst du das Drehbuch. Glaube an deine Träume und lebe sie. Du bist ein Original und keine Kopie. Sei ehrlich zu dir selbst und in deinem Handeln. Entwickle deine Stärken weiter und arbeite an deinen Schwächen. Achte auf deinen Körper und deine Gefühle. Erkenne und respektiere dabei deine Grenzen und die Grenzen anderer. \\
	\includegraphics[width=2cm]{Ausgaben/Sola24/Grafiken/lupe.jpg} & \textbf{Wag es, nach dem Sinn deines Lebens zu suchen!} \newline
	Mach dich auf den Weg, deinen Glauben an Gott und den Sinn deines Lebens zu finden. Lass die anderen teilhaben an deinen Überzeugungen, aber auch an deinen Zweifeln. Dann bist du auf deiner Suche nie allein. \\
	\includegraphics[width=2cm]{Ausgaben/Sola24/Grafiken/lebensstil.png} & \textbf{Wag es, deinen eigenen Lebensstil zu finden!} \newline
	Nutze deine Freiheit, dich auszuprobieren und finde deinen persönlichen Stil. Du wirst schnell merken, was dir gefällt und zu dir passt. Hab den Mut, du selbst zu sein und vereinfache deine Ansprüche. Prüfe selbst und entscheide, was du wirklich brauchst. \\
	\includegraphics[width=2cm]{Ausgaben/Sola24/Grafiken/eye.png} & \textbf{Wag es, deine Augen aufzumachen!} \newline
	Wir leben gemeinsam auf der Welt. Nimm Ungerechtigkeit und Intoleranz wahr und nenn sie beim Namen. Beschäftige dich mit dem, was im Leben um dich herum und in der Welt geschieht und hab einen Blick dafür, wo Hilfe nötig ist. Frage lieber einmal mehr als einmal zu wenig, warum etwas so ist, wie es ist. Wenn dir etwas nicht gefällt, dann versuche es zu ändern. \\
	\includegraphics[width=2cm]{Ausgaben/Sola24/Grafiken/Meinung.jpg} & \textbf{Wag es, deine Meinung zu vertreten!} \newline
	Trau dich, deinen Mund aufzumachen und zu deiner Meinung zu stehen. Wenn du den Mut findest, wirst du merken, dass du es kannst. Lerne deine Kritik so zu formulieren, dass du andere dadurch nicht verletzt. Nimm die Kritik anderer an und denke über dein eigenes Verhalten nach. Entscheide dann, ob du etwas an dir ändern möchtest. \\
	\includegraphics[width=2cm]{Ausgaben/Sola24/Grafiken/steps.jpg} & \textbf{Wag es, den nächsten Schritt zu tun!} \newline
	Wage Risiko und Abenteuer. Hab keine Angst einen Fehler zu machen. Überwinde dein Bedürfnis, nach Sicherheit und Perfektion zu streben. Suche nach neuen Wegen und Möglichkeiten. Nimm Schwierigkeiten als Herausforderung an und lerne, mit Rückschlägen fertig zu werden. \\
	\includegraphics[width=2cm]{Ausgaben/Sola24/Grafiken/brush.jpg} & \textbf{Wag es, dein Leben aktiv zu gestalten!} \newline
	Mach deine eigenen Pläne. Tue bewusst, was du tust. Eigeninitiative und Kreativität machen dich zu einem einmaligen Menschen. Es ist immer besser selbst etwas zu tun als nur rumzusitzen und zuzuschauen. Wenn Du selbst aktiv wirst, findest du auch Unterstützung. \\
	\includegraphics[width=2cm]{Ausgaben/Sola24/Grafiken/natur.png} & \textbf{Wag es, dich für die Natur einzusetzen!} \newline
	Mach dich auf, die Vielfalt und Schönheit der Natur kennen zu lernen. Sie ist Teil der Schöpfung. Nutze die Chancen, die dir die Natur an Erlebnissen und Erholung bietet. Lerne so umweltbewusst zu leben, dass alle, die nach dir kommen, die gleiche Vielfalt und Schönheit erleben können wie du. Trete öffentlich und aktiv für den Erhalt der Schöpfung ein. \\
	\end{tabular}
	\newpage