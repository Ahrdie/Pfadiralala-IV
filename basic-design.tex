\usepackage{ifthen}

% Mit oder ohne Bildern kompilieren
\newboolean{pics} %Deklaration
\setboolean{pics}{true}

% Mit oder ohne Schnittmarken kompilieren
\newboolean{cuts} %Deklaration
\setboolean{cuts}{false}

\usepackage[dvips=false,pdftex=false,vtex=false,left=1.2cm,right=1.2cm,
top=0.4cm,bottom=0.5cm,includeheadfoot,
paperwidth=148mm,paperheight=210mm,twoside]{geometry}

% Schnittmarken, auskommentieren für randlose Version
\ifthenelse{\boolean{cuts}}
        {\usepackage[cam,a4,center,dvips]{crop}}{}

% \ifthenelse{\equal{\leftmark}{\rightmark}}
%         {\rightmark}
%         {From \rightmark\ until \leftmark}
% \usepackage[cam,a4,center,dvips]{crop}

% Songs Latex Package
\usepackage[chorded, noshading]{songs}

\newcommand{\songlink}[2]{\hyperlink{#1}{#2}}

% Umlaute
\usepackage[ngerman]{babel}
\usepackage[utf8]{inputenc}

% PDF Seiten Import, Watermarks für Daumenregister und Multicolumn für das Impressum
\usepackage{pdfpages}
\usepackage{watermark}
\usepackage{multicol}

% Hintergrundbild
\usepackage{wallpaper}

%Kein Einzug bei Absatz-Beginn
\setlength{\parindent}{0in} 			

% Helvetica
\usepackage[T1]{fontenc}
\usepackage[scaled]{helvet}

%Überschriften zentrieren
\usepackage[center]{titlesec}

% Header und Footer
\usepackage{fancyhdr}
\pagestyle{fancy}	

\fancyhead{} % clear all header fields
\fancyfoot{} % clear all footer fields

% Seitenzahl bei geraden/linken Seiten nach links/aussen
% Seitenzahl bei ungeraden/rechten Seiten nach rechts/aussen
\fancyfoot[LE,RO]{\thepage}

% Pfadiralala Oben Innen
\fancyhead[LO,RE]{Pfadiralala IV}

% Ignore sectionmarks - own mymarks
\renewcommand{\sectionmark}[1]{}
\newcommand{\mymarks}{%
        \ifthenelse{\equal{\leftmark}{\rightmark}}
                {\rightmark}
                {From \rightmark\ until \leftmark}
        }
\fancyhead[LE,RO]{\mymarks}

%Inhaltsverzeichnisse
% \newindex{titelIndex}{titelIndexF}
% Seitennummer anstatt Liednummer
% \indexsongsas{titelIndex}{\thepage}
 
% Genau eine Spalte für die Lieder
\songcolumns{1}

% Balken Refrain
\setlength{\cbarwidth}{0pt}
% Balken Liedanfang und Ende											
\setlength{\sbarheight}{0pt}											

% keine Nummerierung der Lieder
\nosongnumbers

% Schriftarten für die verschiedenen Teile
% Lieder 
\renewcommand*\familydefault{\sfdefault}
% Überschriften
\renewcommand{\stitlefont}{\sffamily\bf\LARGE\centering}
% Strophen
\renewcommand{\lyricfont}{\sffamily}
% Akkorde
\renewcommand{\printchord}[1]{\sffamily\bf#1}
% Refrain
%\renewcommand{\chorusfont}{\it}
\renewcommand{\everychorus}{\textnote{\bf Refrain}}
% Beschreibungen
%\renewcommand{\scripturefont}{\sffamily\it}
% Inhaltsverzeichnis
\renewcommand{\idxtitlefont}{\sffamily} 
\renewcommand{\idxlyricfont}{\sffamily}


% Darstellung der Meta-Informationen
% Kein Autor unter Titel
\renewcommand{\extendprelude}{}

% Worte und Weise
\newcommand{\wuweise}{}
\newsongkey{wuw}{\def\wuweise{}}
        {\def\wuweise{ Worte und Weise: #1}}

% Melodie
\newcommand{\melodie}{}
\newsongkey{mel}{\def\melodie{}}
        {\def\melodie{ Melodie: #1}}

% Text
\newcommand{\text}{}
\newsongkey{txt}{\def\text{}}
        {\def\text{ Text: #1}}
        
% Album
\newcommand{\album}{}
\newsongkey{alb}{\def\album{}}
        {\def\album{ Album: #1}}
        
% Jahr
\newcommand{\jahrformat}{}
\newsongkey{jahr}{\def\jahrformat{}}
        {\def\jahrformat{, #1}}

% Bock
\newcommand{\bock}{}
\newsongkey{bo}{\def\bock{}}
        {\def\bock{Liederbock: #1 }}

% Pfadiralala I
\newcommand{\pfadi}{}
\newsongkey{pfi}{\def\pfadi{}}
        {\def\pfadi{Pfadiralala I: #1 }}

% Pfadiralala II
\newcommand{\pfadii}{}
\newsongkey{pfii}{\def\pfadii{}}
        {\def\pfadii{Pfadiralala II: #1 }}

% Pfadiralala III
\newcommand{\pfadiii}{}
\newsongkey{pfiii}{\def\pfadiii{}}
        {\def\pfadiii{Pfadiralala III: #1 }}

% Jurtenburg
\newcommand{\jurte}{}
\newsongkey{ju}{\def\jurte{}}
        {\def\jurte{Jurtenburg: #1 }}


 %Nach dem Lied
\renewcommand{\extendpostlude}{
    \normalsize\bf
        \wuweise
        \melodie
        \text
        \album
        \jahrformat
    \par
    \normalsize\it
        \bock
        \pfadi
        \pfadii
        \pfadiii
        \jurte
    \break
    }

% Notennamen
%\notenamesin{A}{B}{C}{D}{E}{F}{G}
%\notenamesout{A}{H}{C}{D}{E}{F}{G}

