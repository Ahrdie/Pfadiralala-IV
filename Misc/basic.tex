\usepackage{ifthen}

% Enables retrieval of ENV variables (https://tex.stackexchange.com/questions/62010/can-i-access-system-environment-variables-from-latex-for-instance-home/62032#62032)
\usepackage{catchfile}
\newcommand{\getenv}[2][]{%
  \CatchFileEdef{\temp}{"|kpsewhich --var-value #2"}{}%
  \if\relax\detokenize{#1}\relax\temp\else\let#1\temp\fi}

\getenv[\PRINT]{PRINT}
\getenv[\PICS]{PICS}
\def\true{true }

\newboolean{pics}
\setboolean{pics}{false}

% Keine Bilder einfügen, wenn ein draft erzeugt wird.
\ifx\PICS\true
	\setboolean{pics}{true}
\fi

% Schnittmarken hinzufügen, wenn ein Druckdokument erzeugt wird.
\ifx\PRINT\true
	\setboolean{pics}{true}
	\usepackage[cam,a4,center,dvips]{crop}
\fi

\usepackage[dvips=false, pdftex=false, vtex=false,
	left=1.2cm, right=1.2cm,
	top=0.4cm, bottom=0.5cm,
	% left=1.5cm, right=1.5cm, % wir-machen-druck rand
	% top=0.7cm, bottom=0.8cm, % wir-machen-druck rand
	includeheadfoot,
	paperwidth=148mm, paperheight=210mm,
	% paperwidth=154mm, paperheight=216mm, % wir-machen-druck rand
	twoside]{geometry}

% Songs Latex Package
\usepackage{hyperref}
\usepackage[chorded, noshading]{Misc/songs}

\PassOptionsToPackage{draft}{graphicx}

% Songs Patch: force space before rep  
\let\songsrep\rep
\renewcommand\rep[1]{~\songsrep{#1}}

% different spacing
\versesep=10pt plus 2pt minus 4pt
\afterpreludeskip=2pt
\beforepostludeskip=2pt

% Umlaute
\usepackage[ngerman]{babel}
\usepackage[utf8]{inputenc}

% PDF Seiten Import, Watermarks für Daumenregister und Multicolumn für das Impressum
\usepackage{pdfpages}
\usepackage{watermark}
\usepackage{multicol}

% Hintergrundbild
\usepackage{wallpaper}

%Kein Einzug bei Absatz-Beginn
\setlength{\parindent}{0in} 			

% Helvetica
\usepackage[T1]{fontenc}
\usepackage[scaled]{helvet}

% Überschriften zentrieren
\usepackage[center]{titlesec}

% Header und Footer
\usepackage{fancyhdr}
\pagestyle{fancy}	

% clear all header & footer fields
\fancyhead{}
\fancyfoot{}

% remove chapter / section namings from headers
\renewcommand\chaptermark[1]{}
\renewcommand\sectionmark[1]{}

% Seitenzahl auf die unteren äußeren Ecken
\fancyfoot[LE,RO]{\thepage}

% Wenn nicht vorhanden: bookname definieren
\providecommand{\bookname}{}
% BuchName Oben Innen
\fancyhead[LO,RE]{\bookname{}}

% Songtitel auf die oberen äußeren Ecken
\fancyhead[LE]{\leftmark}
\fancyhead[RO]{\rightmark}

% \beginsong patchen, damit \songtitle in die fancyhdr marks geschrieben wird
\usepackage{etoolbox}
\makeatletter
\appto\SB@@@beginsong{\markboth{\songtitle}{\songtitle}}
\appto\SB@@@endsong{\markboth{}{}}
\makeatother

% Genau eine Spalte für die Lieder
\songcolumns{1}

% Balken Refrain
\setlength{\cbarwidth}{0pt}
% Balken Liedanfang und Ende											
\setlength{\sbarheight}{0pt}											

% keine Nummerierung der Lieder
\nosongnumbers

% Schriftarten für die verschiedenen Teile
% Lieder 
\renewcommand*\familydefault{\sfdefault}
% Überschriften
\renewcommand{\stitlefont}{\sffamily\bf\LARGE\centering}
% Strophen
\renewcommand{\lyricfont}{\sffamily}
% Akkorde
\renewcommand{\printchord}[1]{\sffamily\bf#1}
% Refrain
%\renewcommand{\chorusfont}{\it}
\renewcommand{\everychorus}{\textnote{\bf Refrain}}
\newcommand{\printchorus}[0]{
	\renewcommand{\everychorus}{\textnote{\bf Refrain (wdh.)}}
	\beginchorus
	\endchorus
	\renewcommand{\everychorus}{\textnote{\bf Refrain}}
}
\newcommand{\repchorus}[1]{
	\renewcommand{\everychorus}{\textnote{\bf Refrain (#1x)}}
	\beginchorus
	\endchorus
	\renewcommand{\everychorus}{\textnote{\bf Refrain}}
}

\newcommand{\interlude}[1]{
	\beginverse*
	{\nolyrics #1}
	\endverse
}
% Beschreibungen
%\renewcommand{\scripturefont}{\sffamily\it}
% Inhaltsverzeichnis
\renewcommand{\idxtitlefont}{\sffamily} 
\renewcommand{\idxlyricfont}{\sffamily}

\newcommand{\capostr}{}
\newsongkey{capo}{\def\capostr{}}
        {\def\capostr{Capo #1}}

% Darstellung der Meta-Informationen
% Kein Autor unter Titel
\renewcommand{\extendprelude}
  {\hfill\normalsize\bf\capostr}

% Worte und Weise
\newcommand{\wuw}{}
\newsongkey{wuw}{\def\wuw{}}
        {\def\wuw{ Worte und Weise: #1}}

% Melodie
\newcommand{\mel}{}
\newsongkey{mel}{\def\mel{}}
        {\def\mel{ Melodie: #1}}

% Text
\newcommand{\txt}{}
\newsongkey{txt}{\def\txt{}}
        {\def\txt{ Text: #1}}
        
% Album
\newcommand{\alb}{}
\newsongkey{alb}{\def\alb{}}
        {\def\alb{ Album: #1}}
        
        
% Lager
\newcommand{\lager}{}
\newsongkey{lager}{\def\lager{}}
        {\def\lager{ #1}}
        
% Jahr
\newcommand{\jahr}{}
\newsongkey{jahr}{\def\jahr{}}
        {\def\jahr{, #1}}
        
% Jahr Text
\newcommand{\txtjahr}{}
\newsongkey{txtjahr}{\def\txtjahr{}}
        {\def\txtjahr{, #1}}
        
% Jahr Melodie
\newcommand{\meljahr}{}
\newsongkey{meljahr}{\def\meljahr{}}
        {\def\meljahr{, #1}}
        
% Originaltonart
\newcommand{\tonart}{}
\newsongkey{tonart}{\def\tonart{}}
        {\def\tonart{ Originaltonart: #1}}

% Bock
\newcommand{\bo}{}
\newsongkey{bo}{\def\bo{}}
        {\def\bo{Liederbock: #1 }}

% Pfadiralala I
\newcommand{\pfi}{}
\newsongkey{pfi}{\def\pfi{}}
        {\def\pfi{Pfadiralala I: #1 }}

% Pfadiralala II
\newcommand{\pfii}{}
\newsongkey{pfii}{\def\pfii{}}
        {\def\pfii{Pfadiralala II: #1 }}

% Pfadiralala III
\newcommand{\pfiii}{}
\newsongkey{pfiii}{\def\pfiii{}}
        {\def\pfiii{Pfadiralala III: #1 }}

% Jurtenburg
\newcommand{\ju}{}
\newsongkey{ju}{\def\ju{}}
        {\def\ju{Jurtenburg: #1 }}
        
% Das Grüne (VCP Stamm Parzival, Niedernhausen)
\newcommand{\gruen}{}
\newsongkey{gruen}{\def\gruen{}}
        {\def\gruen{Das Grüne: #1 }}
        
% Kinder-Schoko-Songs (VCP Region Wetterau)
\newcommand{\kssiv}{}
\newsongkey{kssiv}{\def\kssiv{}}
        {\def\kssiv{Kinder-Schoko-Songs IV: #1 }}
        
% Die singende Runde (VCP Stamm Ingelheim)
\newcommand{\siru}{}
\newsongkey{siru}{\def\siru{}}
        {\def\siru{Die singende Runde: #1 }}

% Das Biest (VCP Gau Alt-Burgund)
\newcommand{\biest}{}
\newsongkey{biest}{\def\biest{}}
        {\def\biest{Das Biest: #1 }}

% EG (Evangelisches Gesangbuch)
\newcommand{\eg}{}
\newsongkey{eg}{\def\eg{}}
        {\def\egplus{Evangelisches Gesangbuch: #1 }}

% EGplus (Evangelisches Gesangbuch plus)
\newcommand{\egplus}{}
\newsongkey{egplus}{\def\egplus{}}
        {\def\egplus{EGplus: #1 }}


% Nach dem Lied 
\renewcommand{\extendpostlude}{
    {\normalsize\bf
        \mel
        \meljahr
        \txt
        \txtjahr
        \wuw
        \alb
        \jahr
        \tonart
    \par}
    {\small
        \lager
    \par}
    {\normalsize\it
        \bo
        \pfi
        \pfii
        \pfiii
        \ju
        \gruen
        \kssiv
        \siru
        \biest
        \egplus
    \break}
}

% Deutsches H statt englisches B (wichtig für transponierte Noten)
\notenames{A}{H}{C}{D}{E}{F}{G}

% Thumb-Indexes using LaTeX
\usepackage{fp}
\newcommand{\setthumb}[2]{
    \FPeval{\result}{#2}                % Convert the input parameter to a number (no idea why fp is needed for this)
    \setlength{\unitlength}{7.78mm}     % Set the distance of the individual thumb markers
    \begin{intersong}
    \leftwatermark{
        \put(0, -\result) {
            \setlength{\unitlength}{1pt}
            \put(-63, 0)    {\rule{60pt}{25pt}}
            \put(-28, 0)    {\Huge \color{white} \makebox(25, 25){#1}}
        }
    }
    \rightwatermark{
        \put(0, -\result) {
            \setlength{\unitlength}{1pt}
            \put(356, 0)    {\rule{60pt}{25pt}}
            \put(356, 0)    {\Huge \color{white} \makebox(25, 25){#1}}
        }
    }
    \end{intersong}
}

\newcommand{\nothumb}{
    \leftwatermark{}
    \rightwatermark{}
}

\usepackage{wrapfig}

\usepackage{tabularx}
