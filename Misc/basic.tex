\usepackage{ifthen}

% Enables retrieval of ENV variables (https://tex.stackexchange.com/questions/62010/can-i-access-system-environment-variables-from-latex-for-instance-home/62032#62032)
\usepackage{catchfile}
\newcommand{\getenv}[2][]{%
  \CatchFileEdef{\temp}{"|kpsewhich --var-value #2"}{}%
  \if\relax\detokenize{#1}\relax\temp\else\let#1\temp\fi}

\getenv[\PRINT]{PRINT}
\getenv[\PICS]{PICS}
\def\true{true }

\newboolean{pics}
\setboolean{pics}{false}

% Keine Bilder einfügen, wenn ein draft erzeugt wird.
\ifx\PICS\true
	\setboolean{pics}{true}
\fi

% Schnittmarken hinzufügen, wenn ein Druckdokument erzeugt wird.
\ifx\PRINT\true
	\setboolean{pics}{true}
	\usepackage[cam,a4,center,dvips]{crop}
\fi

\usepackage[dvips=false, pdftex=false, vtex=false,
	left=1.2cm, right=1.2cm,
	top=0.4cm, bottom=0.5cm,
	% left=1.5cm, right=1.5cm, % wir-machen-druck rand
	% top=0.7cm, bottom=0.8cm, % wir-machen-druck rand
	includeheadfoot,
	paperwidth=148mm, paperheight=210mm,
	% paperwidth=154mm, paperheight=216mm, % wir-machen-druck rand
	twoside]{geometry}

% Songs Latex Package
\usepackage{hyperref}
\usepackage[chorded, noshading]{Misc/songs}

\usepackage[generate, ps2eps]{abc}

\newcommand{\songlink}[2]{\hyperlink{#1}{#2}}

% Umlaute
\usepackage[ngerman]{babel}
\usepackage[utf8]{inputenc}

% PDF Seiten Import, Watermarks für Daumenregister und Multicolumn für das Impressum
\usepackage{pdfpages}
\usepackage{watermark}
\usepackage{multicol}

% Hintergrundbild
\usepackage{wallpaper}

%Kein Einzug bei Absatz-Beginn
\setlength{\parindent}{0in} 			

% Helvetica
\usepackage[T1]{fontenc}
\usepackage[scaled]{helvet}

% Überschriften zentrieren
\usepackage[center]{titlesec}

% Header und Footer
\usepackage{fancyhdr}
\pagestyle{fancy}	

% clear all header & footer fields
\fancyhead{}
\fancyfoot{}

% remove section namings from headers
\renewcommand\sectionmark[1]{}

% Seitenzahl auf die unteren äußeren Ecken
\fancyfoot[LE,RO]{\thepage}

% Songtitel auf die oberen äußeren Ecken
\fancyhead[LE]{\leftmark}
\fancyhead[RO]{\rightmark}

% "Pfadiralala IV" Oben Innen
\fancyhead[LO,RE]{Pfadiralala IV}

% \beginsong patchen, damit \songtitle in die fancyhdr marks geschrieben wird
\usepackage{etoolbox}
\makeatletter
\appto\SB@@@beginsong{\markboth{\songtitle}{\songtitle}}
\appto\SB@@@endsong{\markboth{}{}}
\makeatother

% Genau eine Spalte für die Lieder
\songcolumns{1}

% Balken Refrain
\setlength{\cbarwidth}{0pt}
% Balken Liedanfang und Ende											
\setlength{\sbarheight}{0pt}											

% keine Nummerierung der Lieder
\nosongnumbers

% Schriftarten für die verschiedenen Teile
% Lieder 
\renewcommand*\familydefault{\sfdefault}
% Überschriften
\renewcommand{\stitlefont}{\sffamily\bf\LARGE\centering}
% Strophen
\renewcommand{\lyricfont}{\sffamily}
% Akkorde
\renewcommand{\printchord}[1]{\sffamily\bf#1}
% Refrain
%\renewcommand{\chorusfont}{\it}
\renewcommand{\everychorus}{\textnote{\bf Refrain}}
\newcommand{\printchorus}[0]{
	\renewcommand{\everychorus}{\textnote{\bf Refrain (wdh.)}}
	\beginchorus
	\endchorus
	\renewcommand{\everychorus}{\textnote{\bf Refrain}}
}
\newcommand{\repchorus}[1]{
	\renewcommand{\everychorus}{\textnote{\bf Refrain (#1x)}}
	\beginchorus
	\endchorus
	\renewcommand{\everychorus}{\textnote{\bf Refrain}}
}

\newcommand{\interlude}[1]{
	\beginverse*
	{\nolyrics #1}
	\endverse
}
% Beschreibungen
%\renewcommand{\scripturefont}{\sffamily\it}
% Inhaltsverzeichnis
\renewcommand{\idxtitlefont}{\sffamily} 
\renewcommand{\idxlyricfont}{\sffamily}

% Darstellung der Meta-Informationen
% Kein Autor unter Titel
\renewcommand{\extendprelude}{}

% Worte und Weise
\newcommand{\wuweise}{}
\newsongkey{wuw}{\def\wuweise{}}
        {\def\wuweise{ Worte und Weise: #1}}

% Melodie
\newcommand{\melodie}{}
\newsongkey{mel}{\def\melodie{}}
        {\def\melodie{ Melodie: #1}}

% Text
\newcommand{\text}{}
\newsongkey{txt}{\def\text{}}
        {\def\text{ Text: #1}}
        
% Album
\newcommand{\album}{}
\newsongkey{alb}{\def\album{}}
        {\def\album{ Album: #1}}
        
% Jahr
\newcommand{\jahrformat}{}
\newsongkey{jahr}{\def\jahrformat{}}
        {\def\jahrformat{, #1}}

% Bock
\newcommand{\bock}{}
\newsongkey{bo}{\def\bock{}}
        {\def\bock{Liederbock: #1 }}

% Pfadiralala I
\newcommand{\pfadi}{}
\newsongkey{pfi}{\def\pfadi{}}
        {\def\pfadi{Pfadiralala I: #1 }}

% Pfadiralala II
\newcommand{\pfadii}{}
\newsongkey{pfii}{\def\pfadii{}}
        {\def\pfadii{Pfadiralala II: #1 }}

% Pfadiralala III
\newcommand{\pfadiii}{}
\newsongkey{pfiii}{\def\pfadiii{}}
        {\def\pfadiii{Pfadiralala III: #1 }}

% Jurtenburg
\newcommand{\jurte}{}
\newsongkey{ju}{\def\jurte{}}
        {\def\jurte{Jurtenburg: #1 }}


 %Nach dem Lied
\renewcommand{\extendpostlude}{
    \normalsize\bf
        \wuweise
        \melodie
        \text
        \album
        \jahrformat
    \par
    \normalsize\it
        \bock
        \pfadi
        \pfadii
        \pfadiii
        \jurte
    \break
    }

% Deutsches H statt englisches B (wichtig für transponierte Noten)
\notenames{A}{H}{C}{D}{E}{F}{G}

