\beginsong{Matte un Kees}[wuw={Fäägmeel}, index={Schu längst leure Glocke}, jahr=1989, alb={Lieder fier Noochbersch}]

\beginverse 
Schu \[D]längst leure Glocke, die Braut kimmt net bei,
mim \[G]Schleier is \[D]irgendwoas \[A]net in de Reih',
die \[D]Hauptsach se kimmt noch und werds aach se spät...
\endverse

\beginchorus
Loas \[G]erschtemol \[D]Matte ge\[A]währn - irchend\[D]wann gibt's aach Kees.
Irgendwei gibt's \[A]irgendwann aachemol \[D]Kees.
\endchorus

\beginverse
Mit ^zwelf horre sich schon im Spiegel beguckt,
un's ^Kinn hoat aach ^als so ver^dächtich gejuckt,
er ^wollt sich rasiern owwers woar naut se seh,
\endverse

\renewcommand{\everychorus}{\textnote{\bf Refrain (wdh.)}}
\beginchorus
\endchorus

\beginverse
Die ^Mudder is schwanger, die Klaa waas beschaad,
und ^stoppt sich zwa ^Kisserche ^inner des Klaad.
"Wann's ^Baby kimmt", säht se, "braach ich aach e Schees"...
\endverse

\beginchorus
\endchorus

\beginverse
Beim ^Fußball do leihe mir fünf-null se rick,
un e^weil is groad ^Halbzeit, do ^hu mer noch Glück,
Wei ^git doas jetz weirer, denkt jeder nervös... 
\endverse

\beginchorus
\endchorus

\beginverse
Im ^Winter, wanns kaalt wird, wirds manchmol aach weiß,
die ^Kinn hu de ^leibst rächt viel ^Schnee un viel Eis,
un ^eas aach bis Christdoag naut weißes se see...
\endverse

\renewcommand{\everychorus}{\textnote{\bf Refrain \rep{2}}}
\beginchorus
\endchorus

\endsong
