\beginsong{Viva La Pappa col Pomodoro}[
    wuw={Lina Wertmüller, Nino Rota}, 
    jahr={1963}, 
    txt={Letizia Gamerro, Jonas Höchst (Übersetzung)},
]

\beginverse
% Ver\[E]gangene Ge\[A]schichte und tausende Berichte
% erzähl'n von leeren Bäuchen und von Revolu\[E]tion.
% Und weil wir einmal kämpften um den Hunger zu stillen
% und gegen große Villen gibt's doppelte Por\[A]tion.
\endverse
\centering\includegraphics[width=1\textwidth]{Noten/VivaLaPappa.pdf}

\beginchorus
\[E7]Vi\[A]va la pa-pa pa-ppa col po-po-po-po-po-po-pomo\[E7]doro! 
Viva la pa-pa pa-ppa che è un capo-po-po-po-po-la\[A]voro!
Vi\[A7]va la pa pa-\[D]ppa pa-pa col \[E7]po-po-pomo\[A]dor.
\endchorus

\beginverse
Der \[E7]Magen grummelt \[Am]leise, wir wollen Aufstand üben, 
Verschwörung wegen Speisen, weg mit dem Direk\[E7]tor!
Die Suppe ist jetzt fertig, und alle singen heiter,
für uns geht es gut weiter: la pappa al pomo\[A]dor!
% und für uns geht es weiter mit der Revolu\[A]tion!
\endverse

\printchorus

\endsong

\beginscripture{}
\ifthenelse{\boolean{pics}}{
    \begin{wrapfigure}{R}{0.3\textwidth}
        \vspace{5cm}
    \end{wrapfigure}
}{}
Dieses Lied, das junge Heimkinder wie eine Revolutionshymne singen, richtet sich gegen eine zu strenge Erziehung und gegen ekliges Essen. Es ist der Soundtrack aus der Fernsehserie ''Il giornalino di Gian Burrasca'' aus dem Jahr 1964, bei dem Lina Wertmüller Regie geführt hat und die bekannte Schlagersängerin Rita Pavone in der Hauptrolle zu sehen war.
\endscripture
