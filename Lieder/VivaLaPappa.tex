\beginsong{Viva La Pappa col Pomodoro}[wuw={Rita Pavone}, jahr={1963}, txt={Jonas Höchst}]

\beginverse
% Ver\[E]gangene Ge\[A]schichte und tausende Berichte
% erzähl'n von leeren Bäuchen und von Revolu\[E]tion.
% Und weil wir einmal kämpften um den Hunger zu stillen
% und gegen große Villen gibt's doppelte Por\[A]tion.
\endverse
\centering\includegraphics[width=1\textwidth]{Noten/VivaLaPappa.pdf}

\beginchorus
\[E7]Vi\[A]va la pa-pa pa-ppa col po-po-po-po-po-po-pomo\[E7]doro! 
Viva la pa-pa pa-ppa che è un capo-po-po-po-po-la\[A]voro!
Vi\[A7]va la pa pa-\[D]ppa pa-pa col \[E7]po-po-pomo\[A]dor.
\endchorus

\beginverse
Der \[E7]Magen grummelt \[Am]leise, wir wollen Aufstand üben, 
Verschwörung wegen Speisen, und gegen die Frak\[E7]tion!
Die Suppe ist jetzt fertig, nun singen alle heiter,
und für uns geht es weiter mit der Revolu\[A]tion!
\endverse

\printchorus

\endsong
