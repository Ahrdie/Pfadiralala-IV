\beginsong{Gute Nacht, Freunde}[
    wuw={Reinhard Mey}, 
    jahr={1972}, 
    pfii={181}, 
    pfi={102}, 
    pfii={181}, 
    siru={92}, 
    biest={552},
]


% \beginchorus
% \[A] Gute Nacht, \[Hm]Freunde,\[E] es wird Zeit für mich zu \[A]geh'n. \[D]
% Was ich noch zu sagen \[C#m]hätte, dauert eine Ziga\[Hm]rette
% \[E] und ein letztes Glas im \[A]steh'n.
% \endchorus
%
% \beginverse\memorize
% Für den Tag, für die \[Hm]Nacht unter eurem Dach habt \[E]Dank,
% für den Platz an eurem \[A]Tisch, für jedes Glas, das ich trank.
% Für den Teller, den Ihr \[Hm]mir zu den euren \[E]stellt,
% als sei selbstver\[A]ständlicher \[D]nichts auf der \[E]Welt.
% \endverse

\beginverse
\endverse
\includegraphics[page=1]{Noten/GuteNachtFreunde.pdf}

\beginverse
Habt Dank für die \[Hm]Zeit, die ich mit euch verplaudert \[E]hab',
und für eure Ge\[A]duld, wenn's mehr als eine Meinung gab,
dafür, dass ihr nie \[Hm]fragt, wann ich komm' oder \[E]geh',
für die stets off'ne \[A]Tür, in \[D]der ich jetzt \[E]steh'.
\endverse

\beginchorus
\[A] Gute Nacht, \[Hm]Freunde,\[E] es wird Zeit für mich zu \[A]geh'n. \[D]
Was ich noch zu sagen \[C#m]hätte, dauert eine Ziga\[Hm]rette
\[E] und ein letztes Glas im \[A]steh'n.
\endchorus

\beginverse
Für die Freiheit, ^die als steter Gast bei euch ^wohnt,
habt Dank, dass ihr nie ^fragt, was es bringt, ob es lohnt.
Vielleicht liegt es da^ran, dass man von draußen ^meint,
dass in euren ^Fenstern das ^Licht wärmer ^scheint.
\endverse

\printchorus

\endsong

\beginscripture{}
Mey schrieb das Lied unter dem Pseudonym ''Alfons Yondraschek'' für das Gesangsduo Inga und Wolf, das damit im Vorentscheid zum Eurovision Song Contest 1972 den vierten Platz belegte. Es kam in den deutschen Charts auf Rang 22.
\endscripture
