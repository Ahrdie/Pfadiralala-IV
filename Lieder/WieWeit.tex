\beginsong{Wie weit ist es bis zum Horizont?}[wuw={Knorkator, 2003}]

\markboth{\songtitle}{\songtitle}

\beginverse\memorize
\[Em]Wie weit mag es sein bis zum Horizont?\[D] \[C]Diese Frage \[Am]will \[G]ich euch \[D]be\[D#]antwor\[Em]ten.
\[Em]Steh ich auf der Welt, meines Blickes Strahl\[D] \[C]trifft die Erde \[Am]als Tan\[G]gente\[D] am \[D#]Hori\[Em]zont.
\[Dm]Dann um neunzig \[Am]Grad \[Dm]bis zum Erdmittel\[Am]punkt \[Gm]hab ich den Erdra\[F]di\[D#]us.
\[Dm]Nun zurück zu meinem \[Am]Kopf: \[Dm]Radius plus ein \[Am]Mensch \[Gm]gibt es ein rechtwinkliges \[F]Drei\[D#]eck.
\endverse

\beginchorus
\[Dm]Wie weit \[C]ist es \[B&]bis zum \[A]Hori\[Gm]zont? \[F#]
\endchorus

\beginverse
^Diese Entfernung ist A, der Radius ist B,^ ^Mittelpunkt bis ^Kopf ^ist die ^Sei^te ^C.
^Nehmen wir den Satz des Pythagoras:^ ^A-Quadrat Plus ^B-Qua^drat ist ^C-^Qua^drat.
^Stellen wir dieses ^um, ^so errechnet sich ^A ^aus der Wurzel der ^Diffe^renz
^zwischen C zum Qua^drat ^minus B zum Qua^drat, ^Fehlen nur noch die ^Zah^len.
\endverse

\beginchorus
\[Dm]Wie weit \[C]ist es \[B&]bis zum \[A]Hori\[Gm]zont? \[F#]
\endchorus

\beginverse
^Der Erdradius B misst in etwa Sechs^-mil^lionen-dreihundert^achtund^siebzig-^tau^send^meter.
^C gleich Sechs-millionen-dreihundertacht^und^siebzig-tausend ^und ^einen ^Me^ter^siebzig.
^Bildet man die Qua^drate, ^so ist deren Diffe^renz ^einundzwanzig Mil^lio^nen
^sechshundertachzig^tausend, ^nun die Wurzel da^raus: ^viertausendsechshundert^fünzig ^Meter.
\endverse

\beginchorus
\lrep \[Dm]So weit \[C]ist es \[B&]bis zum \[A]Hori\[Gm]zont. \rrep
\endchorus

\endsong

\beginscripture{}
\endscripture

\begin{intersong}

\end{intersong}