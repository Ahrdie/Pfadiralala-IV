\beginsong{Sixteen Tons}[wuw={Traditional / Merle Travis}, jahr=1947]

\beginverse\memorize
\[Am]Some people say a man is \[Dm]made out of \[E]mud,
\[Am]a poor man's made out of \[Dm]muscle 'n \[E]blood.
\[Am]Muscle an blood and \[Dm]skin an bone,
\[Am]a mind that's weak and a \[E]body that's \[Am]strong...
\endverse

\beginchorus
\[Am]You load a sixteen \[G]tons an \[F]what do you \[E]get?
\[Am]Another day \[G]older and \[F]deeper in \[E]debt,
\[Am]Saint Peter don't you call me, 'cause I \[Dm]can't go,
\[Am]I owe my soul to the \[E]company \[Am]store. \[G] \[F] \[Em]
\endchorus


\beginverse
\[Am]I was born on morning, when the \[Dm]sun didn't sh\[E]ine,
\[Am]picked up my shovel and I \[Dm]went to the \[E]mine,
\[Am]loaded sixteen tons of \[Dm]number nine coal,
\[Am]and the straw boss said, ''Well, \[E]bless my \[Am]soul!''
\endverse

\printchorus

\beginverse
^I was born one morning, it was ^drizzlin' ^rain,
^fightin' and trouble are my ^middle ^name,
^I was raised in a canebreak by an ^old mama lion,
^can't no hightoned woman ^make me walk the ^line.
\endverse

\printchorus

\beginverse
^If you hear me a-comin, you ^better step a^side,
^a lotta men didn't, ^a lotta men di^ed.
^With one fist of iron an the ^other of steel,
^if the right one don't a-get you then the ^left one ^will...
\endverse

\printchorus

\endsong

\beginscripture{}
Sixteen Tons ist ein sozialkritischer Country-/Folk-Song, der 1947 von Merle Travis veröffentlicht wurde. 1955 machte ihn Tennessee Ernie Ford zum Nummer-eins-Hit in den Country- und Popcharts sowie zum Millionenseller. Das Stück beschreibt das Leben in US-amerikanischen Kohlegruben etwa zur Zeit des Zweiten Weltkriegs.
\endscripture