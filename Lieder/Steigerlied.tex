\beginsong{Steigerlied}[
    wuw={Traditionell (Ursprünge im 16. Jahrhundert)}, 
]
    
\transpose{+5}

\beginverse
\endverse
\includegraphics[page=1]{Noten/Steigerlied.pdf}

\beginverse\memorize
\[G]Hat's \[D]ange\[G]zünd't, 's \[G]wirft \[D7]seinen \[G]Schein,
und da\[G]mit so fahren \[D7]wir bei der Nacht,
und da\[C]mit so fahren \[G]wir bei der Nacht
ins \[Em]Berg\[D]werk \[G-C-G]ein, \[Am]ins \[G]Berg\[D7]werk \[G]ein.
\endverse

\beginverse
^Ins ^Bergwerk ^ein, ^wo die ^Bergleut' ^sein,
die da ^graben das Silber und das ^Gold bei der Nacht,
die da ^graben das Silber und das ^Gold bei der Nacht
aus ^Fels^ge^stein, ^aus ^Fels^ge^stein.
\endverse

\beginverse
Der ^eine ^gräbt das ^Silber, der ^andere ^gräbt das ^Gold.
Und dem ^schwarzbraunen Mägde^lein bei der Nacht,
und dem ^schwarzbraunen Mägde^lein bei der Nacht
dem ^sei'n ^sie ^hold, ^dem ^sei'n ^sie ^hold.
\endverse

\beginverse
^A^de, A^de! ^Herz^liebste ^mein!
Und da ^drunten in dem tiefen, finstren ^Schacht bei der Nacht,
und da ^drunten in dem tiefen, finstren ^Schacht bei der Nacht,
da ^denk ^ich ^dein, ^da ^denk ^ich ^dein.
\endverse

\beginverse
^Und ^kehr‘ ich ^Heim ^zur ^Liebsten ^mein,
dann er^schallet des Bergmanns ^Gruß bei der Nacht:
dann er^schallet des Bergmanns ^Gruß bei der Nacht:
Glück ^auf, ^Glück ^auf! ^Glück ^auf, ^Glück ^auf!
\endverse

\beginverse
^Wir ^Bergleut' ^sein, ^kreuz^brave ^Leut',
denn wir ^tragen das Leder vor dem ^Arsch bei der Nacht,
denn wir ^tragen das Leder vor dem ^Arsch bei der Nacht
und ^sau^fen ^Schnaps, ^und ^sau^fen ^Schnaps!
\endverse

\endsong

\beginscripture{}
Das Steigerlied ist die Hymne der Bergleute. Im Lied wird die Hoffnung besungen, nach schwerer Arbeit wieder unversehrt nach Hause zurück zu kommen. 

Die Ursprünge des Liedes gehen in das 16. Jahrhundert zurück, heute wird es zudem häufig bei den Heimspielen von Fußballvereinen gesungen, die in der Tradition des Bergbaus stehen, wie dem FC Erzgebirge Aue, Rot-Weiss Essen und dem FC Schalke 04.
\endscripture