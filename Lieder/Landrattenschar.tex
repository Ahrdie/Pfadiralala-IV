\beginsong{Landrattenschar}[
    wuw={mama (Till Schöllhammer), VCP Stamm Grafen von Eberstein (Worms)}, 
    jahr={2012},
    biest={586}, 
    index={Keine Schwiele an der Hand}, 
]

\beginverse
% \[H7]Keine \[Em]Schwiele an der Hand, und das \[C]Tau noch unbekannt,
% rührt der \[G]Seegang noch in Bauch und \[H7]Sinn,
% sucht der \[Em]Fuß ständig Halt, und der \[C]Wind scheint noch kalt,
% zieht die \[H7]junge Crew erstmals da\[Em]hin.
\endverse
\includegraphics[draft=false, page=1]{Noten/Landrattenschar.pdf}

\beginverse
''\[H7]Klar zur \[Em]Wende?'' schallt ‘s vom Heck, doch der \[C]Himmel schlägt leck,
Hält das \[G]Ölzeug dem Regen nicht \[H7]stand.
Trotzdem \[Em]heißt es zum Tau, ist der \[C]Magen auch flau,
meilen\[G]weit scheint das rettende \[H7]Land!
\endverse

% Bridge
\beginverse*
% Doch mit \[Am]jedem Tag, den der \[D]Bug so zerteilt,
% schwämmt die \[G]See uns Gewissheit ins \[H7]Blut,
% die wir ^so Mal zu Mal mit den ^Winden geeilt,
% gibt die ^Sonne den nötigen ^Mut!
\endverse
\includegraphics[draft=false, page=1]{Noten/Landrattenschar-Bridge.pdf}

\beginchorus
So hisst die \[Em]Segel, oh elende \[C]Landrattenschar,
denn der \[G]Seegang ruft weit uns hi\[H7]naus.
Haltet das \[Em]Ruder auf Kurs, und das \[C]Vordeck macht klar,
wahrt den \[H7]Kopf trotz Getos‘ und Ge\[Em]braus.
\endchorus

\beginverse
^Plötzlich ^Stille an Bord, Wind und ^Regen sind fort,
Sonnen^schein lädt zum Sprung von der ^Schanz!
Tauchen ^unter den Kiel, sorgen^los wildes Spiel,
zwischen ^Himmel und der Wellen ^Tanz.
\endverse

% Bridge
\beginverse*
Und am \[Am]Abend schallen Lieder, hin\[D]aus auf die See,
künden \[G]vom Wind dem längst wir ver\[H7]trau‘n.
Sichert der ^Anker den Schlaf hält das ^Schiff in Lee,
wiegen die ^Wellen uns von Traum zu ^Traum. 
\endverse

\beginchorus
\lrep Längst sind die \[Em]Segel gehisst und die \[C]Schoten bemannt,
warten die \[G]Winde auf unseren \[H7]Ruf!
Ein neuer \[Em]Kurs ist bestimmt und alles \[C]geht Hand in Hand,
gieren \[G]wir mit dem Segel nach \[H7]Luv! \rrep
\endchorus

\endsong
