\beginsong{Cecilia Lind}[
    mel={Traditional, Bearbeitung: Little Pink, 2021},
    txt={Cornelis Vreeswijk, 1966. Deutsche Umdichtung in Kreisen des Überbündischen Frauenforums},
    txtjahr={1995},
    index={Vom Tanzplatz ertönt}, 
]

\beginverse
% Vom \[Dm]Tanzplatz ertönt wieder \[E]Geige und \[Am]Bass,
% der \[Dm]Vollmond, er \[G]leuchtet, als \[C]sei er aus \[E]Glas.
% Es \[Am]tanzt Alt und \[C]Jung, ob \[Am]Oma, ob \[E]Kind
% und \[Am]mittendrin \[Dm]dreht sich Ce\[E]cilia \[Am]Lind.
\endverse
\includegraphics[draft=false, page=1]{Noten/CeciliaLindPink.pdf}

\beginverse
Sie \[Dm7]lacht und sie strahlt und sie \[E7]tanzt nicht al\[A7]lein,
man \[Dm7]fragt sich: Wer \[G]mag wohl der \[C]andere \[E7]sein?
Man \[Am]munkelt und \[C]rätselt und \[Am]starrt plötzlich \[E7]hin:
Da \[Dm7]tanzt eine \[Dm7]Frau mit Ce\[E7]cilia \[Fmaj7]Lind!
\endverse

\beginverse
Ver^sunken, umschlungen und ^auch ziemlich ^dicht,
die ^Blicke der ^Leute be^merken sie ^nicht.
Was ^flüstert die ^And're ^leis wie der ^Wind
und ^zärtlich ins ^Ohr von Ce^cilia ^Lind?
\endverse

\beginverse
''Ce^cilia, du Schöne, es ^war mir längst ^klar,
als ich ^dich im Wen^Do-Kurs zum ^ersten Mal ^sah.\newpage
Ja, ^war'n denn die ^Frauen um ^dich alle ^blind?
Mein ^Glück, denn ich ^sah dich, Ce^cilia ^Lind.''
\endverse

\beginverse
Am ^Rande hock Fredrik mit ^einem Glas ^Wein.
''Was ^fehlt denn?'', fragt ^Lasse, ''Wa^rum so al^lein?''
''Da ^drüben die b^eiden, ja ^schau doch mal ^hin,
das ^war doch mal ^meine Ce^cilia ^Lind.''
\endverse

\beginverse
''Mensch ^Fredrik,'', sagt Lasse, ''das ^geht doch nicht ^an,
geh' ^rüber und ^zeig' ihnen: ^Du bist ein ^Mann!''
Doch ^das war ein ^Fehler, ^zeigt sich ge^schwind,
denn ^sie kann jetzt ^WenDo - Ce^cilia ^Lind.
\endverse

\beginverse
Die ^Sterne wander'n, die ^Stunden ver^geh'n,
der ^Vollmond, er ^leuchtet, als s^ei nichts ge^scheh'n.
Und ^mitten im ^Trubel, mit ^schmerzendem ^Kinn,
hockt ^Fredrik und ^grollt mit Ce^cilia ^Lind.
\endverse

\beginverse
Die ^Geigen schweigen, der ^Morgen ist ^nah,
^einzelne ^gehen und ^auch manches ^Paar.
Zwei ^Frauenge^stalten, weißt ^du, wer sie ^sind?
Ver^schwanden im ^Haus von Ce^cilia ^Lind...
\endverse

\endsong

\beginscripture{}
Das irische Original ''Monday Morning'' wurde unter anderem von Peter, Paul and Mary aufgenommen und hat die mit dem 16. Geburtstag der Protagonistin verbundene Hochzeit zum Inhalt.
Im schwedischen Text von Cornelis Vreeswijk lernt Cecilia Lind den älteren Fredrik Åkare beim Tanz kennen. Am Anfang ihrer Romanze stehen moralische Bedenken, ''aber die Liebe ist blind'' und die beiden küssen sich.
In der deutschen Übersetzung aus Kreisen des Überbündischen Frauenforums nimmt den schwedischen Text auf und verpasst ihm eine Überraschende Wendung. Cecilia hat in der Zwischenzeit einen WenDo-Kurs (Selbstverteidigung für Frauen) besucht und überwältigt Fredrik als dieser mit seinen Annäherungsversuchen ihre Grenzen überschreitet. 
\endscripture