\beginsong{Von guten Mächten}[txt={Dietrich Bonhoeffer}, jahr=1944, kssiv={314}, pfi={55}, siru={242}]

% \beginverse\memorize
% Von \[C]guten Mächten \[G]treu und still \[Am]umgeben
% \[F]behütet und ge\[C]tröstet wunder\[Dm]bar. \[G]
% So \[C]will ich diese \[G]Tage mit euch \[Am]leben
% und \[F]mit euch gehen \[C]in ein \[G]neues \[C]Jahr
% \endverse

\beginverse
\endverse
\centering\includegraphics[width=1\textwidth, page=1]{Noten/VonGutenMaechten.pdf}
\centering\includegraphics[width=1\textwidth, page=2]{Noten/VonGutenMaechten.pdf}


\beginverse\memorize
Noch \[D]will das alte \[A]unsre Herzen \[Hm]quälen,
noch \[G]drückt uns böser \[Em]Tage schwere \[A]Last. \[A7]
Ach \[D]Herr, gib unsern \[A]aufgeschreckten \[Hm]Seelen
das \[G]Heil, für das du \[D]uns ge\[A]schaffen \[D]hast.
\endverse

\beginchorus
Von \[D]guten Mächten \[A]wunderbar ge\[Hm]borgen, \[D7]
erwarten \[G]wir getrost \[Em] was kommen \[A7]mag \[A]
Gott \[D]ist mit uns am \[A]Abend und am \[Hm]Morgen \[D/f#]
und \[Em]ganz gewiss an \[A]jedem neuen \[D]Tag.
\endchorus

\beginverse
Und ^reichst du uns den ^schweren Kelch, den ^bittern
des ^Leids, gefüllt bis ^an den höchsten ^Rand, ^
So ^nehmen wir ihn ^dankbar ohne ^Zittern
aus ^deiner guten ^und gel^iebten ^Hand.
\endverse

\printchorus

\beginverse
Doch ^willst du uns noch ^einmal Freude ^schenken,
an ^dieser Welt und ^ihrer Sonne ^Glanz. ^
Dann ^wolln wir des Verg^angenen ge^denken,
und ^dann gehört dir ^unser ^Leben ^ganz.
\endverse

\beginchorus
Von \[D]guten Mächten \[A]wunderbar ge\[Hm]borgen, \[D7]
erwarten \[G]wir getrost \[Em] was kommen \[A7]mag \[A]
Gott \[D]ist mit uns am \[A]Abend und am \[Hm]Morgen \[D/f#]
und \[Em]ganz gewiss an \[A]jedem neuen \[D]Tag.
\endchorus

\beginverse
Lass ^warm und hell die ^Kerzen heut' er^flammen,
die ^du in unsre ^Dunkelheit ge^bracht. ^
Führ, ^wenn es sein kann, ^wieder uns zu^sammen.
Wir ^wissen es, dein ^Licht scheint ^in der ^Nacht.
\endverse

\printchorus

\beginverse
Wenn ^sich die Stille ^nun tief um uns ^breitet,
so ^lass uns hören ^jenen vollen ^Klang ^
der ^Welt, die unsicht^bar sich um uns ^weitet,
all ^deiner Kinder ^hohen ^Lobge^sang.
\endverse

\printchorus

\endsong

\beginscripture{}
Der evangelische Theologe und NS-Widerstandskämpfer Dietrich Bonhoeffer verfasste den Text im Dezember 1944 nach über eineinhalb Jahren in der Gestapo-Haft. Es ist der letzte erhaltene theologische Text vor seiner Hinrichtung am 9. April 1945.
\endscripture
