\beginsong{500 Jahre Musik}[
    wuw={Nils Berkey, Jonas Höchst, 2011}, 
    index={Vor gut 500 Jahren},
]

\beginverse
\[Em]Vor gut fünf\[Csus2]hundert Jahren
fing der \[Em]erste von uns an die \[Csus2]anderen zu plagen
\[Em]damals hatten sie noch \[Csus2]nicht so lange Haare
\[Em]aber sie sangen und \[Csus2]das ist doch das wahre
\endverse

\beginchorus
\lrep \[G]Sie sangen \[D/f#]laut,\[Am] sie sangen schön
\[G] und kamen \[D/f#]bis, in \[Am]alle Höh'n. \rrep
\endchorus

\beginverse
\[Em]Heut' sitz' ich hier und \[Csus2]mach' Musik
\[Em]und denk' daran, wie's \[Csus2]früher lief:
\[Em]Ausgebeutet \[Csus2]für den Krieg,
ein \[Em]Lied für den to\[Csus2]talen Sieg.
\endverse

\beginchorus
\lrep \[G]Sie grölten \[D/f#]laut,\[Am] sie schrie'n zusamm'n
\[G] und taten \[D/f#]das, was \[Am]sonst keiner kann. \rrep
\endchorus

\beginverse
^Doch es gab auch Jahre, da ^war das ganze anders,
^da gab's Konzerte mit 'nem ^friedlichen Anlass,
Ge^sang für den Frieden und ^gegen all das,
was uns ^unterdrückt mit ^Prügel und Hass.
\endverse

\beginchorus
\lrep \[G]Sie sang'n für \[D/f#]Peace,\[Am] sie sang'n für Liebe
\[G] und demons\[D/f#]trierten für den \[Am]Weltfrieden. \rrep
\endchorus

\beginchorus
\[G] Lalala\[D/f#]la,\[Am] lalalalala.
\[G] Lalala\[D/f#]la, lala\[Am]laaa.
\[G] Lalala\[D/f#]la,\[Am] lalalalala.
\[G]Lalalala\[D/f#]la, \[Am]Weltfrieden! \[Em]
\endchorus

\endsong

\beginscripture{}
Das Lied ist entstanden aus Langeweile und überschüssiger Kreativität am Vorbereitungslager 2011 des Stammes Graf-Folke-Bernadotte Fulda.
\endscripture