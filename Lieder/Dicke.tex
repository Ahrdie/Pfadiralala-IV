\beginsong{Dicke}[
    wuw={Marius Müller-Westernhagen}, 
    alb={Mit Pfefferminz bin ich dein Prinz}, 
    jahr={1978}, 
    pfii={156}, 
    index={Ich bin froh, dass ich kein Dicker bin},
]

\beginchorus
\[Am]Ich bin ich froh, dass ich kein Dicker bin, 
denn \[G]dick sein ist 'ne Quälerei.
\[F]Ich bin froh, dass ich so'n dürrer Hering bin, 
denn \[E]dünn bedeutet frei zu sein.
\endchorus

\beginverse\memorize
Mit \[Am]Dicken macht man gerne Späße, \[G]Dicke haben Atemnot.
Für \[F]Dicke gibt's nichts anzuzieh'n, \[E]Dicke sind zu dick zum flieh'n.
\[Am]Dicke haben schrecklich dicke Beine, \[G]Dicke ha'm 'n Doppelkinn.
\[F]Dicke schwitzen wie die Schweine, \[E]stopfen, fressen in sich 'rin.
\endverse

\beginchorus
Und darum ^bin ich froh, dass ich kein Dicker bin, 
denn ^dick sein ist 'ne Quälerei.
^Ich bin froh, dass ich so'n dürrer Hering bin, 
denn ^dünn bedeutet frei zu sein.
\endchorus

\beginverse
^Dicke haben Blähungen, ^Dicke ham' 'en dicken Po.
Und ^von den ganzen Abführmitteln ^rennen Dicke oft auf's Klo.
\endverse

\printchorus
\interlude{Zwischenspiel: \[Am G F E] \rep{2}}

\beginverse
^Dicke müssen ständig fasten, 
da^mit sie nicht noch dicker werd'n.
Und ^ham' sie endlich zehn Pfund abgenommen, 
ja dann ^kann man es noch nicht mahl seh'n.
^Dicke ham's auch schrecklich schwer mit Fraun', 
denn ^Dicke sind nicht angesagt.
D'rum ^müssen Dicke auch Karriere machen, 
mit ^Kohle ist man auch als Dicker gefragt.
\endverse

\printchorus
\interlude{Zwischenspiel: \[Am G F E] \rep{2}}

\beginverse
^Dicke, Dicke, Dicke, Dicke. ^Dicke, Dicke, Dicke, Dicke.
^Dicke, Dicke, Dicke, Dicke. ^Dicke, Dicke, Dicke, Dicke.
\endverse

\interlude{Zwischenspiel: \[Am G F E] \rep{2} ''Na, du fette Sau?''}

\endsong
