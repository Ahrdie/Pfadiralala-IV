\beginsong{Rock My Soul}[wuw={Traditional (Gospel)}, kssiv={323}, pfi={135}]

\beginchorus
\endchorus
\centering\includegraphics[width=1\textwidth]{Noten/RockMySoul.pdf}	 

% \beginverse\memorize
% \[D]Rock my soul in the bosom of Abraham,
% \[A7]rock my soul in the bosom of Abraham,
% \[D]rock my soul in the bosom of Abraham,
% \[A7]oh, rock my soul!
% \endverse

\beginverse
\[D]So high, I can't get over it,
\[A7]so low, I can't get under it,
\[D]so wide, I can't get round of it,
\[A7]oh, rock my \[D]soul!
\endverse

\beginverse
^Rock my soul...
^Rock my soul...
^Rock my soul...
^Oh, rock my ^soul!
\endverse

\endsong

\beginscripture{}
Der Schoß Abrahams spielt auf ''Das Gleichnis vom reichen Mann und vom armen Lazarus'' an; eine biblische Erzählung aus dem neuen Testament (Lukas 16,19-31). Auf den Schoß eines Großvaters (sinnbildlich für den Stammvater Abrahams) dürfen eigentlich nur Kinder. Und dann sind es meist die Enkel und die nahe Verwandtschaft, die ein inniges Vertrauensverhältnis genießen. So drückt das Lied den tiefen Wunsch aus, eine solches Gottvertrauen auch selbst zu erleben.
\endscripture
