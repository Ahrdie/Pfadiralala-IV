\beginsong{Gori Kaseki}[
    wuw={trenk (Alo Hamm), Zugvogel Deutscher Fahrtenbund}, 
    biest={580}, 
    tonspur={358}, 
    siru={135}, 
    index={In Gori Kaseki am Rande der Straße}, 
    jahr={1943},
]

% \beginverse
% In \[Em]Gori Kaseki am \[G]Rande der Straße liegt \[Am]einsam ein Knabe, der \[H7]regt sich nimmer\[Em]mehr.
% Ge\[Em]fährliche Straßen, un\[G]heimliche Weiten, da\[Am]rüber die Wolken sind \[H7]wie ein Geister\[Em]heer.
% \endverse
% \beginchorus
% \lrep \[G]Zwei\[C]tausend \[G]Rei\[Em]ter \[H7]federn he\[Em]ran, ja heran,
% \[Am]was sind da\[Em]gegen einhundert\[H7]fünfundachzig \[Em]Mann? \rrep
% \endchorus

\beginverse
\endverse
\includegraphics[draft=false, page=1]{Noten/GoriKaseki.pdf}

\beginverse\memorize
In \[Em]Gori Kaseki sind \[G]nur noch Ruinen, 
aus \[Am]Hütten, die blieben, kein \[H7]Laut, sie stehen \[Em]leer.
Ge\[Em]flohen die Dörfler, die \[G]Hunde, die Katzen, 
es \[Am]hausen nur Ratten bei \[H7]dem geschmolz'nen \[Em]Heer. 
\endverse

\beginchorus
\lrep \[G]Zwei\[C]tausend \[G]Rei\[Em]ter \[H7]federn he\[Em]ran, ja heran,
\[Am]was sind da\[Em]gegen einhundert\[H7]fünfundachzig \[Em]Mann? \rrep
\endchorus

\beginverse
In ^Gori Kaseki ver^brannt ist die Erde, 
ver^giftet die Brunnen, hier ^fängt die Hölle ^an.
Ge^frorene Brote mit ^Beilen sie teilen 
und ^Schneewasser nur für den ^todgeweihten ^Mann.
\endverse

\printchorus

\beginverse
Erst ^schicken sie Frauen, dann ^Kinder, dann Greise, 
sie ^liegen im Eise, ein ^schneebedeckter ^Wall.
Da^rüber sie sprengen, gleich ^wilden Gesängen, 
mit ^Unrast die Reiter, des ^Todes Va^sall. 
\endverse

\printchorus

\beginverse
In ^Gori Kaseki sind ^alle geblieben, 
Zwei^tausendeinhundert-^fünfundachzig ^Mann.
Vom ^Himmel kam Feuer, da^rin sind sie geblieben, 
Zwei^tausendeinhundert-^fünfundachzig ^Mann.
\endverse

\repchorus{2}

\endsong

\beginscripture{}
Trenk war Soldat im 2. Weltkrieg und war am 23.6.1943 in Gory Kaseki (heute Pobeda, Russland) stationiert und äußerte sich 1950 zum Lied:


''Das Lied von Gori Kaseki ist eigentlich stellvertretend für Situationen dieser Art, wie sie in dem Lied beschrieben werden, wie man sie allgemein damals als Soldat im Ostfeldzug, also in Russland erlebte. Ich muss dazu sagen, dass es in Russland damals tatsächlich noch richtige Soldaten auf Pferden gegeben hat, allerdings waren es in Gori Kaseki keine 2000 Reiter und man soll deshalb gar nicht enttäuscht sein darüber. Es waren Panzerwagen, die uns niedermachten. Aber es ist im Russlandfeldzug auf beiden Seiten noch die Kavallerie in Aktion getreten, vor allen Dingen bei den Russen. Und weil es sich vom Lied her so besser machte, ist es in diese Form gebracht worden.''
\endscripture