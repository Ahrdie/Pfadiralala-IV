\beginsong{Verliebt in Du}[wuw={Lüül}, jahr=2004, alb={Damenbesuch}, index={Jaja es stimmt, ich geb es zu}]


\renewcommand{\everychorus}{\textnote{\bf Anfang}}
\beginchorus
\[C] Jaja es stimmt,\[G] ich geb es zu: Ich bin ver\[Am]liebt!
\[C] Jaja es stimmt,\[G] ich geb es zu: Ich bin ver\[Am]liebt in Du.
\endchorus

\beginverse
Ich \[C]denk an dich, wo Du auch bist, ich \[G]stell mir vor, wie das wohl ist,
wenn \[Am]du mich küsst, wie sich das fühlt, wenn \[F]durchgewühlt dein wildes Haar
auf \[G]nackten Schultern fließt, du süßes Biest.
\endverse

\renewcommand{\everychorus}{\textnote{\bf Refrain}}
\beginchorus
Ich bin ver\[C]liebt\[G]... Ver\[Am]liebt in du!
\[C] Jaja es stimmt, \[G]ich geb es zu: Ich bin ver\[Am]liebt in Du.
\endchorus


\beginverse
Nicht ^wegen deinem kurzen Rock und ^deinen langen Beinen,
ja ^Bock hab ich schon deswegen – ^ne, aber eher deine Lust zu Leben
und deine ^Art zu sein und außerdem dein grüner Blick.
\endverse

\printchorus

\renewcommand{\everychorus}{\textnote{\bf Übergang}}
\beginchorus
Komm \[F]spiel mit deinem Bogen, deinen stolzen Ton so zart und \[E]schön.
so \[F]süß, vertraut und steck ihn mir ins Ohr, in Herz und Bauch
und was da \[E]noch so ist, das mir \[G]schwindlig wird, es stimmt.
\endchorus
\renewcommand{\everychorus}{\textnote{\bf Refrain}}

\printchorus
	
% \beginverse
% Ich ^denk an dich, wo Du auch bist, ich ^stell mir vor, wie das wohl ist,
% wenn ^du mich küsst, wie sich das fühlt, wenn ^durchgewühlt dein wildes Haar
% auf ^nackten Schultern fließt, du süßes Biest.
% \endverse
\beginverse
wie 1.
\endverse

\printchorus

\endsong