\beginsong{Du bist du (Vergiss es nie)}[txt={Jürgen Werth}, mel={Paul Janz, 1977}]

\beginverse\memorize
Vergiss es \[C]nie: Dass du \[Am]lebst, war keine \[Em]eigene Ide\[Am]e, 
und dass du \[G]atmest, \[G7]kein Entschluss von \[C]dir.
Vergiss es \[C]nie: Dass du \[Am]lebst war eines \[Em]anderen Ide\[Am]e, 
und dasd du \[G]atmest, \[G7]sein Geschenk an \[C]dich.
\endverse

\beginchorus
\[C] Du bist ge\[F]wollt kein Kind des \[A]Zufalls, keine \[Dm]Laune der Natur,
ganz \[G]egal ob du dein \[G7]Lebenslied in \[C]Moll singst oder Dur.
Du bist \[Am]ein Gedanke \[E]Gottes, ein geni\[Am]aler noch da\[F]zu!
Du bist \[C]du, das ist der Clou,
ja, der \[G]Clou. Ja, du bist \[C]du.
\endchorus

\beginverse
Vergiss es ^nie: Niemand ^denkt und fühlt und ^handelt so wie ^du,
und niemand ^lächelt so, ^wie du's grad ^tust.        
Vergiss es^ nie: Niemand ^sieht den ^Himmel ganz genau wie ^du
und niemand ^hat je, was ^du weißt, ge^wusst.
\endverse

\printchorus

\beginverse
Vergiss es ^nie: Dein ^Gesicht hat ^niemand sonst auf dieser ^Welt, 
und solche  ^Augen ^hast alleine ^du.       
Vergiss es ^nie: Du bist ^reich, egal, ob ^mit, ob ohne ^Geld,
denn du kannst ^leben!  ^Niemand lebt wie ^du. 
\endverse

\printchorus

\endsong

\beginscripture{}
Im englischen Original ''I got you'' ist sowohl der Text als auch die Melodie von Paul Janz.
\endscripture

\begin{intersong}
\end{intersong}