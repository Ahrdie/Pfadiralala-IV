\beginsong{Near to Banbridge Town (deutsch)}[mel={Musik aus dem Irischen}, meljahr={1726}, txt={Cathal Mac Garvey}, pfii={30}, pfiii={11}, index={Eines Morgens ging}]

\beginverse
Eines \[Em]Morgens ging \[D]ich so für mich \[G]hin \[D]im \[Em]Julisonnen\[D]schein,
Den \[Em]Wiesenpfad, \[D]den \[G]Hang hi\[D]nab kam ein \[Em]schönes Mäg\[D]de\[Em]lein.
Und sie \[G]lacht mich an und ich \[D] freu' mich dran und be\[Em]wund're ihr nussbraunes \[D]Haar.
Einer \[Em]lockenden Fee \[D]kam ich \[G]kaum in die \[D]Näh, ganz ver\[Em]wirrt von dem \[D]nussbraunen \[Em]Haar.
\endverse

\beginchorus
\lrep Oh, from \[G]Bantry Bay up to \[D]Derry Quay and from \[Em]Galway to Dublin \[D]Town, 
No maid \[Em]I've seen \[D]like this \[G]brown col\[D]leen, that I \[Em]met in the Coun\[D]ty \[Em]Down. \rrep 
\endchorus

\beginverse
Doch sie \[Em]ging unbeirrt \[D]ihres Weges, \[G]verwirrt \[D]stand ich \[Em]da und nur eins war mir \[D]klar.
Als ein \[Em]Bauer kam, \[D]sprach ich: '\[G]Lieber \[D]Mann, wer ist \[Em]die mit dem nuss\[D]braunen \[Em]Haar?'
Und der \[G]Mann lacht mich an und mit \[D]Stolz sagt er dann: 'Sie ist die \[Em]Perle von Irlands \[D]Kron'!
Uns're \[Em]Rosie McGann \[D]von den \[G]Ufern des \[D]Boyne ist der \[Em]Stern der Land\[D]schaft \[Em]dort.'
\endverse

\renewcommand{\everychorus}{\textnote{\bf Refrain (wdh.)}}
\beginchorus
\endchorus


\beginverse
Doch ich ^sah sie beim Tanz ^unter'm ^Ernte^kranz eines ^Abends im Sommer^kleid.
Dann mit ^schmeichelndem Blick ^ver^sucht' ich mein ^Glück um das ^Herz meiner nuss^braunen ^Maid.
Geb mein ^Wort dafür, keinen ^Pflug ich führ', wird das ^Eisen vom Rost auch ^braun.
Sitzt an ^meinem Herd ^die, die ^ich be^gehrt, strahlt der ^Stern von Coun^ty ^Down.
\endverse
\beginchorus
\endchorus

\endsong

\beginscripture{}
Das Lied ist die deutsche Version der irischen Ballade 'Near to Banbridge Town' von Cathal Mac Garvey aus dem Jahr 1726. 
\endscripture
