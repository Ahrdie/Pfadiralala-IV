\beginsong{Nacht in Portugal}[wuw={Dimitrie Miron (DPBM Stamm Sperber)}, siru={268}]

\beginverse
\endverse
\centering\includegraphics[width=1\textwidth, page=1]{Noten/NachtInPortugal.pdf}
\centering\includegraphics[width=1\textwidth, page=2]{Noten/NachtInPortugal.pdf}

% \beginverse\memorize
% Wenn die Sonne sich schon \[Am]senkt, geschieht es ein ums and're \[G]Mal,
% werden Lampen aufge\[F]hängt in dem Dorf in Portu\[E]gal.
% Und wie schon in alter \[Am]Zeit kommt ein Jeder ohne \[G]Hatz,
% die Gitarren sind \[F]bereit, schon erfüllt Musik den \[E]Platz.
% \endverse
%
% \beginchorus
% \lrep Und sie beginnen zu \[Am]tanzen, das Dorf bewegt sich im \[G]Ganzen,
% selbst Kinder und \[F]Greise auf eigene Weise, sie drehn sich im \[E]Kreise. \rrep
% \endchorus

\beginverse
In dieser warmen Juli\[Am]nacht zeigen sie ihr Temp'ra\[G]ment,
keine Pause wird ge\[F]macht und sie tanzen wie ent\[E]hemmt.
Sieht ein Fremder dieses \[Am]Bild, bleibt er nicht sehr lang \[G]allein,
bald tanzt er genauso \[F]wild und es wirbeln seine \[E]Bein'.
\endverse

\renewcommand{\everychorus}{\textnote{\bf Refrain 2}}
\beginchorus
\lrep Und so tanzen sie \[Am]immer, bis der erste Sonnen\[G]schimmer,
die Strahlen aus\[F]sendet, die Augen schon blendet und den Tanz be\[E]endet. \rrep 
\endchorus


\renewcommand{\everychorus}{\textnote{\bf Refrain 1 und Zwischenspiel}}
\beginchorus
\lrep Und sie beginnen zu \[Am]tanzen, das Dorf bewegt sich im \[G]Ganzen,
selbst Kinder und \[F]Greise auf eigene Weise, sie drehn sich im \[E]Kreise. \rrep
\endchorus

\endsong
