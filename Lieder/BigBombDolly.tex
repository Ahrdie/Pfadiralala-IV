\beginsong{Big Bomb Dolly}[mel={Klaus Eckhardt}, txt={Fritz Graßhoff}, jahr=1965]

% \beginverse\memorize
% Wir \[Em]kommen mit dem \[H7]Walfischkahn \[C]zweimal im \[D]Jahr \[Em]nach Haus.
% 'ne Wolke Dunst von \[H7]Lebertran, die \[C]weht dem \[D]Pott vor\[Em]aus.
% \[D]Wir \[G]mähen unser \[D]Sauerkraut und \[G]wetzen in die \[D]Stadt, oh ja,
% weil \[Em]die perfekte S\[H7]eemannsbraut uns l\[C]ängst  ger\[D]ochen h\[Em]at.
% \endverse

\beginverse
\endverse
\centering\includegraphics[width=1\textwidth, page=1]{Noten/BigBombDolly.pdf}
\centering\includegraphics[width=1\textwidth, page=2]{Noten/BigBombDolly.pdf}


\beginverse\memorize
Sitzt \[Em]mal der Käpt'n \[H7]bagebrannt in \[C]Schulden \[D]und in \[Em]Gin,
dann gehen wir mit \[H7]Fisch an Hand und \[C]ohne \[D]Piepen \[Em]hin.
\[D]Die \[G]Dame ist nicht \[D]lasterhaft, drum \[G]rechnet sie ge\[D]nau.
Sie \[Em]nimmt für ihre \[H7]Einsatzkraft zwei \[C]Zentner \[D]Kabel\[Em]jau.
\endverse

\beginchorus
Das ist die \[G]Big Bomb \[D]Dolly aus \[G]Dover, ahoi ohe!
Und die hat Sprengstoff \[D]unterm Pull\[G]over, ahoi ohe!
Die macht \[D]Feuer aus der Heuer in der \[G]Pinte an der See,
dass die \[D]Flusen nur so brennen und die \[G]Leute in Calais
nachts die \[Em]Zeitung \[H7]lesen \[Em H7]könn\[Em]en!
\endchorus

\beginverse
Es ^bleibt nicht aus, dass ^unser Pott den ^letzen ^Kai be^zieht.
Im Seemannsheim Zum ^lieben Gott ^gibt's ^nichts auf ^dem Ge^biet.
^Doch ^findet unser ^scharfer Blick das ^ferne Leuchttsig^nal.
Denn ^unten brennt ja ^noch zum Glück das ^Feuer am Ka^nal!
\endverse

\endsong


\beginscripture{}
\endscripture

\begin{intersong}

\end{intersong}