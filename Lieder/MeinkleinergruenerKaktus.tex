\beginsong{Mein kleiner grüner Kaktus}[wuw={Comedian Harmonists}, jahr={1934}, pfii={140}, kssiv={108}, index={Blumen im Garten}]

\beginverse
\[A]Blumen im \[E7]Garten,\[A] so zwanzig \[E7]Arten\[A] von Rosen, Tulpen und Nar\[F#7]zissen,
\[Hm]leisten sich \[F#7]heute\[Hm] die feinen \[F#7]Leute. \[H7]Das will ich alles gar nicht \[E7]wissen.
\endverse

\beginchorus
Mein \[A]kleiner grüner Kaktus steht draußen am \[E]Balkon. Hollari, hollari, hollar\[A]o!
Was \[A]brauch' ich rote Rosen, was brauch' ich roten \[E]Mohn? Hollari, hollari, hollar\[A]o!
Und \[D]wenn ein Bösewicht was \[A]Ungezognes spricht,
dann \[H7]hol' ich meinen Kaktus und der \[E7]sticht, sticht, sticht.
Mein \[A]kleiner grüner Kaktus steht draußen am \[E]Balkon. Hollari, hollari, hollar\[A]o!
\endchorus

\beginverse
^Man find' gew^öhnlich^ die Frauen ^ähnlich ^den Blumen, die sie gerne ^tragen.
^Doch ich sag' ^täglich:^ 'Das ist nicht ^möglich.'^ Was sollen die Leut' sonst von mir ^sagen?
\endverse

\renewcommand{\everychorus}{\textnote{\bf Refrain (wdh.)}}
\beginchorus
\endchorus

\renewcommand{\everychorus}{\textnote{\bf Refrain}}
\beginverse
^Heute um ^viere^ klopft's an die ^Türe. ^Nanu, Besuch so früh am ^Tage? 
^Es war Herr ^Krause,^ vom Nachbar^hause, ^er sagt: 'Verzeihen sie, wenn ich ^frage:'
\endverse

\beginchorus
Sie \[A]hab'n doch einen Kaktus auf ihrem kleinen \[E]Balkon. Hollari, hollari, hollar\[A]o! 
Der \[A]fiel soeben runter, was halten sie da\[E]von? Hollari, hollari, hollar\[A]o! 
Er \[D]fiel mir auf's Gesicht, ob's \[A]glauben oder nicht.
Jetzt \[H7]weiß ich, dass ihr kleiner grüner \[E]Kaktus sticht. 
Be\[A]wahr'n sie ihren Kaktus gefälligst anders\[E]wo, hollari, hollari, hollar\[A]o! 
\endchorus

\endsong
