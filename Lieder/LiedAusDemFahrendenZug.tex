\beginsong{Lied, aus dem fahrenden Zug zu singen}[wuw={Kurt Demmler}, index={Der Zug fährt auf stählernen Gleisen}]

\beginverse
Der \[Em]Zug fährt auf \[Am]stählernen \[H7]Gleisen, die \[Em]haben wir \[Am]selber ge\[H7]legt, 
dass \[Am]sie auf den \[Em]endlosen Reisen ins \[D]Morgen die \[G]Richtung uns \[Em]weisen
und \[Am]dass unser \[H7]Zug sich be\[Em]wegt.
\endverse

\beginchorus
\[Am]Denn wir müssen alle \[Em]weiterkommen, \[Am]und da dürfen wir nicht \[Em]zaghaft sein. 
Jedes \[Am]Ziel, kaum erreicht, ist schon \[Em]wieder weggeschwommen. \[Am]Also,\[H7] heizt \[Em]ein!
\endchorus

\beginverse
Der ^Zug nimmt auch ^mit all die ^Feigen, die ^meinen, man ^zahlt heut nicht ^mehr. 
Die ^lassen wir ^kurz mal aussteigen, nur ^kurz, und um ^ihnen zu ^zeigen: 
Sch^wer läuft sich's dem ^Zug hinter^her. 
\endverse

\beginverse
Der ^Zug macht viel ^Rauch und Ge^heule, und ^nachts fährt er ^mit Funken^flug.
Da ^grämt sich nur ^immer die Eule, die ^Zeiten der ^klapprigen ^Gäule und 
^Rindviecher ^sind nun ^genug. 
\endverse

\beginverse
Und ^doch führt der ^Zug aus den ^Zeiten der ^Väter manch ^großes ^Gepäck. 
Es ^soll in den ^Wiesen und Weiten der ^Zukunft Er^innerung be^reiten und 
^zeigen: wir ^kommen vom ^Fleck.
\endverse

\beginverse
Der ^Zug jagt den ^glücklichsten ^Träumen der ^Menschen mit ^Macht hinter^her, 
jagt ^nach noch in ^luftlosen Räumen des ^Alls, keine ^Stund' zu ver^säumen, und 
^nähert ^sich mehr und ^mehr.
\endverse

\endsong
