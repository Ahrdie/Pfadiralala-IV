\beginsong{Lied, aus dem fahrenden Zug zu singen}[wuw={Kurt Demmler}, index={Der Zug fährt auf stählernen Gleisen}, siru={49}, biest={528}]

% \beginverse
% Der \[Em]Zug fährt auf \[Am]stählernen \[H7]Gleisen, die \[Em]haben wir \[Am]selber ge\[H7]legt,
% dass \[Am]sie auf den \[Em]endlosen Reisen ins \[D]Morgen die \[G]Richtung uns \[Em]weisen
% und \[Am]dass unser \[H7]Zug sich be\[Em]wegt.
% \endverse
%
% \beginchorus
% \[Am]Denn wir müssen alle \[Em]weiterkommen, \[Am]und da dürfen wir nicht \[Em]zaghaft sein.
% Jedes \[Am]Ziel, kaum erreicht, ist schon \[Em]wieder weggeschwommen. \[Am]Also,\[H7] heizt \[Em]ein!
% \endchorus

\beginverse
\endverse
\includegraphics[page=1]{Noten/LiedAusDemFahrendenZug.pdf}

\beginverse
Der \[Em]Zug nimmt auch \[Am]mit all die \[H7]Feigen, die \[Em]meinen, man \[Am]zahlt heut nicht \[H7]mehr. 
Die \[Am]lassen wir \[Em]kurz mal aussteigen, nur \[C]kurz, und um \[G]ihnen zu \[Em]zeigen: 
Sch\[Am]wer läuft sich's dem \[H7]Zug hinter\[Em]her. 
\endverse

\beginchorus
\[Am]Denn wir müssen alle \[Em]weiterkommen, \[Am]und da dürfen wir nicht \[Em]zaghaft sein. 
Jedes \[Am]Ziel, kaum erreicht, ist schon \[Em]wieder weggeschwommen. \[Am]Also,\[H7] heizt \[Em]ein!
\endchorus

\beginverse
Der ^Zug macht viel ^Rauch und Ge^heule, und ^nachts fährt er ^mit Funken^flug.
Da ^grämt sich nur ^immer die Eule, die ^Zeiten der ^klapprigen ^Gäule und 
^Rindviecher ^sind nun ^genug. 
\endverse

\printchorus

\beginverse
Und ^doch führt der ^Zug aus den ^Zeiten der ^Väter manch ^großes ^Gepäck. 
Es ^soll in den ^Wiesen und Weiten der ^Zukunft Er^innerung be^reiten und 
^zeigen: Wir ^kommen vom ^Fleck.
\endverse

\printchorus

\beginverse
Der ^Zug jagt den ^glücklichsten ^Träumen der ^Menschen mit ^Macht hinter^her, 
jagt ^nach noch in ^luftlosen Räumen des ^Alls, keine ^Stund' zu ver^säumen, und 
^nähert ^sich mehr und ^mehr.
\endverse

\printchorus

\endsong
