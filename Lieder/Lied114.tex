\beginsong{Lauda to si}[wuw={nach dem Sonnengesang des Franz von Assisi}, pfii={128}, index={Sei gepriesen}]

\markboth{\songtitle}{\songtitle}

\beginverse
Sei ge\[G]priesen, du hast die Welt erschaffen.
Sei ge\[Em]priesen für Sonne, Mond und Sterne.
Sei ge\[C]priesen für Meer und Kontinente.
\endverse

\beginchorus
Sei ge\[D]priesen, denn du bist wunderbar, Herr!
\[G]Lauda to si, o mio Signore, \[Em]lauda to si, o mio Signore,
\[C]Lauda to si, o mio Signore, \[D]lauda to si, o mio Signor...
\endchorus

\beginverse
Sei ge^priesen für Licht und Dunkelheiten.
Sei ge^priesen für Nächte und für Tage.
Sei ge^priesen für Jahre und Sekunden. 
\endverse

\beginverse
Sei ge^priesen für Wolken, Wind und Regen.
Sei ge^priesen, Du lässt die Quellen springen.
Sei ge^priesen, Du lässt die Felder reifen.
\endverse

% \beginverse
% Sei ge^priesen für Deine hohen Berge.
% Sei ge^priesen für Fels und Wald und Täler.
% Sei ge^priesen für Deiner Bäume Schatten.
% \endverse

\beginverse
Sei ge^priesen, Du lässt die Vögel singen.
Sei ge^priesen, du lässt die Fische spielen.
Sei ge^priesen für alle deine Tiere.
\endverse

% \beginverse
% Sei ge^priesen, denn Du, Herr, schufst den Menschen!
% Sei ge^priesen, er ist Bild Deiner Liebe!
% Sei ge^priesen für jedes Volk der Erde!
% \endverse
%
% \beginverse
% Sei ge^priesen, Du selbst bist Mensch geworden!
% Sei ge^priesen für Jesus, unser'n Bruder!
% Sei ge^priesen, wir tragen seinen Namen!
% \endverse
%
% \beginverse
% Sei ge^priesen, er hat zu uns gesprochen!
% Sei ge^priesen, er ist für uns gestorben!
% Sei ge^priesen, er ist vom Tod erstanden!
% \endverse
%
% \beginverse
% Sei ge^priesen, oh Herr, für Tod und Leben!
% Sei ge^priesen, Du öffnest uns die Zukunft!
% Sei ge^priesen, in Ewigkeit gepriesen!
% \endverse

\endsong

\beginscripture

\endscripture

\begin{intersong}

\end{intersong}