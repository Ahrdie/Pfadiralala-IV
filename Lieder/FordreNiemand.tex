\beginsong{Fordre Niemand}[mel={Erich Schmeckenbecher}, txt={Deutsches Handwerkerlied um 1840},  index={Fordre niemand mein Schicksal zu hören}, bo=151]

\beginverse\memorize
\[E]Fordre niemand, mein \[H7]Schicksal zu \[E]hö\[A]ren,
von \[E]euch allen, die \[H7]ihr in Arbeit \[E]steht.
\[E]Ja, wohl könnte ich \[H7]Meister be\[E]schwö\[A]ren,
es wär \[E]doch bis \[H7]morgen schon zu \[E]spät.
Aus der \[A]Wanderschaft lustigen \[E]Ta\[H7]-\[E]gen,
setz' ich \[A]Kleider und Reisegeld \[H7]zu.
Und so \[E]hab ich denn nun \[H7]weiter nichts zu \[E]tra\[A]gen,
als mein \[E]Rock und mein \[H7]Stock und die \[E]Schuh.
\endverse

\beginverse
Keine ^Hoffnung ist ^Wahrheit ^gewor^den,
selbst in ^Schlesien war ^alles be^setzt.
Als ich ^reiste über ^Frankfurt nach ^Nor^den,
ward ich ^stets von Gen^darmen ge^hetzt.
Von Stet^tin aus nach Hause ge^schr^ie^ben,
ging ich ^dennoch Berlin erst noch ^zu,
und so ^ist mir denn nun ^weiter nichts ge^blie^ben
als der ^Rock und der ^Stock und die ^Schuh‘.
\endverse

\beginverse
In der ^Heimat darf ^ich mich nicht ^zei^gen,
denn da^hin ist das ^Geld und der ^Rock.
Lasst ^mich meinen ^Namen ver^schwei^gen,
denn sonst ^droht mir ein ^knotiger ^Stock.
Statt in ^Betten, in Wäldern ge^be^t^tet,
wo ich ^hatte nur wenige ^Ruh.
Und so ^hab ich in der ^Ferne nichts ge^ret^tet,
als die ^Hosen und zer^rissene ^Schuh.
\endverse

\endsong
\beginscripture{}
Das Lied ist vermutlich eine Parodie des gleichnamigen Liedes von Karl von Holtei. Diese Version beschreibt das Schicksal der Wandergesellen im 19. Jahrhundert.
\endscripture
