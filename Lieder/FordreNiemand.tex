\beginsong{Ford're Niemand}[
    mel={aus dem Französischen}, 
    txt={von fahrenden Handwerksgesellen},
    txtjahr={1843}, 
    gruen={167}, 
    pfii={74}, 
    siru={84},
    bo={150},
]

% \beginverse\memorize
% \[E]Ford're niemand, mein \[H7]Schicksal zu \[E]hö\[A]ren,
% von \[E]euch allen, die \[H7]ihr in Arbeit \[E]steht.
% \[E]Ja, wohl könnte ich \[H7]Meister be\[E]schwö\[A]ren,
% es \[E]wär doch bis \[H7]morgen schon zu \[E]spät.
% Aus der \[A]Wanderschaft lustigen \[E]Ta\[H7]-\[E]gen,
% setz' ich \[A]Kleider und Reisegeld \[H7]zu.
% Und so \[E]hab ich denn nun \[H7]weiter nichts zu \[E]tra\[A]gen,
% als mein \[E]Rock und mein \[H7]Stock und die \[E]Schuh.
% \endverse

\beginverse
\endverse
\includegraphics[page=1]{Noten/FordreNiemand.pdf}

\beginverse
Keine \[E]Hoffnung ist \[H7]Wahrheit \[E]gewor\[A]den,
selbst in \[E]Schlesien war \[H7]alles be\[E]setzt.
Als ich \[E]reiste über \[H7]Frankfurt nach \[E]Nor\[A]den,
ward ich \[E]stets von Gen\[H7]darmen ge\[E]hetzt.
Von Stet\[A]tin aus nach Hause ge\[E]schr\[H7]ie\[E]ben,
ging ich \[E]dennoch Berlin erst noch \[H7]zu,
und so \[E]ist mir denn nun \[H7]weiter nichts ge\[E]blie\[A]ben
als der \[E]Rock und der \[H7]Stock und die \[E]Schuh‘.
\endverse

\beginverse
In der ^Heimat darf ^ich mich nicht ^zei^gen,
denn da^hin ist das ^Geld und der ^Rock.
Lasst ^mich meinen ^Namen ver^schwei^gen,
denn sonst ^droht mir ein ^knotiger ^Stock.
Statt in ^Betten, in Wäldern ge^be^t^tet,
wo ich ^hatte nur wenige ^Ruh.
Und so ^hab ich in der ^Ferne nichts ge^ret^tet,
als die ^Hosen und zer^rissene ^Schuh.
\endverse

\endsong
\beginscripture{}
Das Lied stammt aus dem 1826 in Berlin uraufgeführten Singspiel ''Der alte Feldherr'' von Karl von Holtei. Die Melodie geht weiter zurück auf das französische ''D'un héros que la France revére''. 
Die vorliegende Version stammt von fahrenden Handwerksgesellen und wurde bis 1843 mündlich überliefert.
\endscripture
