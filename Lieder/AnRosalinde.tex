\beginsong{An Rosalinde}[
    mel={Wolfgang Jehn},
    txt={Margarete Jehn},
]

\beginverse
% \[C]Komm', o \[F]komm', o \[G]komm', Rosa\[C]linde, \[F]dunkel \[C]ist das \[G]Haus ohne \[C]dich.
% \[C]Ach, du \[F]fehlst uns \[G]so, Rosa\[C]linde, \[F]jeder \[C]sitzt al\[G]leine für \[C]sich.
% \[Am]Selbst der \[G]Zeisig \[F]hört auf zu \[C]singen \[F]und das \[C]Pendel der \[G]großen Uhr
% \[C]hört auf, \[F]hin und \[G]her zu \[C]schwingen, \[F]Rosa\[C]linde, \[G]wann kommst du \[C]nur!
\endverse
\includegraphics[page=1]{Noten/AnRosalinde.pdf}

\beginverse\memorize
\[C]Stark und \[F]braun sind \[G]deine \[C]Beine, \[F]und dein \[C]Rock ist \[G]bunt wie ein \[C]Traum.
\[C]Viele \[F]Spritzer, \[G]dreckige \[C]kleine, \[F]äugen \[C]unten \[G]an seinem \[C]Saum.
\[Am]Komm ins \[G]Haus, mach \[F]große \[C]Pfützen\[F] mit dem \[C]Schirm und \[G]mit den Schuh’n.
\[C]Lass uns \[F]nicht auf dem \[G]Trockenen \[C]sitzen, \[F]Rosa\[C]linde, das \[G]kannst du nicht \[C]tun!
\endverse

\beginverse
^Jörg und ^Gustav ^haben ge^schrieben: ''^Unser ^Schiff kommt ^ohne uns ^an.
^Wir sind ^in Aus^tralien ge^blieben, ^weil man ^da was ver^dienen ^kann.
^Seid nicht ^traurig, ^grüßt Rosa^linde ^und den ^Hafen ^und Onkel Jan
^und die ^olle ^Anker^winde ^vor dem ^Seemanns^heim neben^an!''
\endverse

\beginverse
^Komm', o k^omm', o ^komm', Rosa^linde, ^sag uns, ^wo Aus^tralien ^liegt,
^ob man ^da Ko^alas und ^Kängu^rus und ^zahme ^Schlangen ^sieht.
^Da, die ^Tür fliegt ^auf, und sie ^ist es, ^lacht und ^schüttelt den ^Regen ab,
^schmiert ein ^Käse^brot und ^isst es ^und leckt ^sich die ^Finger ^ab.
\endverse

\endsong

\beginscripture{}
% https://www.weser-kurier.de/region/wuemme-zeitung_artikel,-selbst-der-zeisig-hoert-auf-zu-singen-_arid,1579183.html
% http://zurueckauflos.com/564/
% https://www.weser-kurier.de/region_artikel,-Wo-ist-mein-Zuhause-wenn-nicht-hier-_arid,122662.html
Das Lied stammt aus der Feder von Margarete und Wolfgang Jehn, die es im Künstlerdorf Worpswede an viele Menschen weitertrugen, so auch an Eckart Strate, der als ''Dünensänger'' auf Spiekeroog viele Menschen zum Singen begeistert. Sein Sohn und Revolverheld-Sänger Johannes Strate nahm das Lied gemeinsam mit ihm für seine 2011 erschienenen Solo-Platte auf.
\endscripture