\beginsong{Those where the Days}[
    txt={Mary Hopkin, 1968}, 
    mel={Boris Fomin, ca. 1917}, 
    index={Once upon a time there was a tavern}, 
    kssiv={222}, 
    pfiii={109}, 
    pfi={322}, 
    siru={188},
    tonart={Em},
]

\transpose{+5}
\beginverse\memorize
\[Em]Once upon a time there was a tavern, \[E7]where we used to raise a glass or \[Am]two.
Remember how we laughed away the \[Em]hours, \[F#7]dreaming of the great things we would \[H7]do.  
\endverse

\beginchorus
Those were the \[Em]days, my friend, we thought they'd \[Am]never end,  
we'd sing and \[D]dance forever and a \[G]day,  
we'd live the \[Am]life we'd choose, we'd fight and \[Em]never lose,  
for we were \[H7]young and sure to have \[Em]our way.
\endchorus
\interlude{Da-dai-da-\[Em]da da-dai, da-dai-da-\[Am]da da-dai. Da-da-da-\[H7]da, da da da, da-da-\[Em]daa.}

\beginverse
^Then the busy years went rushing by us, we ^lost our starry notions on the ^way.
If by chance I'd see you in the ^tavern, we'd ^smile at one another and we'd ^say.
\endverse

\printchorus

\beginverse
^Just tonight I stood before the tavern. ^Nothing seemed the way it used to ^be.
In the glass I saw a strange re^flection. ^Was that lonely woman really ^me?
\endverse

\printchorus

\beginverse
^Through the door there came familiar laughter, 
I ^saw your face and heard you call my ^name.
Oh, my friend, we're older but no ^wiser, 
for ^in our hearts the dreams are still the ^same.
\endverse

% \printchorus

\endsong

% \beginscripture{}
% Die Musik stammt von dem russischem Lied ''Dorogoi dlinnoju'' von Boris Fomim. Die walisische Sängerin Mary Hopkin gewann mit der englischen Übersetzung einen Fernseh-Talentwettbewerb und wurde darauf hin von Paul McCartney zum Vorsingen eingeladen. Dieser produzierte den Song darauf hin mit dem neu gegründeten Label ''Apple Records''.
% \endscripture
