\beginsong{So ist Versöhnung}[txt={Jürgen Werth}, mel={Johannes Nitsch}, jahr={1988}, index={Wie ein Fest nach langer Trauer}]

\beginverse
% Wie ein \[Em]Fest nach langer \[Hm]Trauer, wie ein \[C]Feuer in der \[Em]Nacht,
% ein off'nes Tor in einer \[Hm]Mauer, für die \[C]Sonne \[D]aufge\[G]macht.
% Wie ein \[Am]Brief nach langem \[D]Schweigen, wie ein \[G]unverhoffter \[C]Gruß,
% wie ein \[Am]Blatt an toten \[Hm]Zweigen, ein ich-\[C]mag-dich-\[D]trotzdem \[Em]Kuss.
\endverse
\centering\includegraphics[width=1\textwidth]{Noten/SoIstVersoehnung.pdf}

\beginverse\memorize
Wie ein \[Em]Regen in der \[Hm]Wüste, frischer \[C]Tau auf dürrem \[Em]Land,
Heimatklänge für Ver\[Hm]misste, alte \[C]Feinde, \[D]Hand in \[G]Hand.
Wie ein \[Am]Schlüssel im Ge\[D]fängnis, wie in \[G]Seenot Land in \[C]Sicht.
Wie ein \[Am]Weg aus der Be\[Hm]drängnis wie ein \[C]strahlen\[D]des Ge\[Em]sicht.
\endverse

\beginchorus
So ist Ver\[D]söhn\[G]ung. So muss der \[D]wahre Friede \[G]sein. 
So ist Ver\[D]söhn\[Em]ung. So ist Ver\[C]geben \[D]und Verz\[Em]eih'n. 
\endchorus

\beginverse
Wie ein ^Wort von toten ^Lippen, wie ein ^Blick, der Hoffnung ^weckt,
wie ein Licht auf steilen ^Klippen, wie ein ^Erdteil, neu ^ent^deckt.
Wie der ^Frühling, wie der ^Morgen, wie ein ^Lied, wie ein Ge^dicht,
wie das ^Leben, wie die ^Liebe, wie Gott ^selbst, das ^wahre ^Licht.
\endverse 

\repchorus{2}

\endsong
