\beginsong{Goldener Reiter}[
    wuw={Joachim Witt}, 
    alb={Silberblick}, 
    jahr={1980}, 
    index={An der Umgehungsstraße},
]

% \beginverse
% \[Em] An der Um\[G]gehungsstraße
% \[D] kurz vor den Mauern unserer Stadt
% \[Em] steht eine \[G]Nervenklinik,
% \[D] wie sie noch keiner gesehen hat.
% \endverse
% \beginverse
% ^ Sie hat das ^Fassungsvermögen
% ^ sämtlicher Einkaufszentren der Stadt.
% ^ Geh'n dir die ^Nerven durch,
% ^ wirst du noch verrückter gemacht.
% \endverse
%
% \beginchorus
% \[Em]Hey, hey, \[G]hey, ich war der goldene \[D]Reiter.
% \[Em]Hey, hey, \[G]hey, ich bin ein Kind dieser \[D]Stadt.
% \[Em]Hey, hey, \[G]hey, ich war so hoch auf der \[D]Leiter,
% Dann fiel ich \[Em]ab,\[G] ja dann fiel ich \[D]ab.
% \endchorus
\beginverse
\endverse
\beginverse
\endverse
\includegraphics[draft=false, page=1]{Noten/GoldenerReiter.pdf}


\beginverse
\[Em] Auf meiner \[G]Fahrt in die Klinik
\[D] sah ich noch einmal die Lichter der Stadt.
\[Em] Sie brannten wie \[G]Feuer in meinen Augen, 
\[D] ich fühlte mich einsam und unendlich schlapp.
\endverse

\beginchorus
\[Em]Hey, hey, \[G]hey, ich war der goldene \[D]Reiter.
\[Em]Hey, hey, \[G]hey, ich bin ein Kind dieser \[D]Stadt.
\[Em]Hey, hey, \[G]hey, ich war so hoch auf der \[D]Leiter,
Dann fiel ich \[Em]ab,\[G] ja dann fiel ich \[D]ab.
\endchorus 

\beginverse
^ Sicherheits^notsignale,
^ Lebensbedrohliche Schizophrenie:
^ Neue Be^handlungszentren 
^ bekämpfen die wirklichen Ursachen nie.
\endverse

\repchorus{2}

\endsong

\beginscripture{}
Das Lied handelt von einem Menschen, der offenbar wegen schwerer Schizophrenie in eine psychiatrische Klinik eingewiesen worden ist. Im Jahr 2016 sagte Joachim Witt in einem Interview, dass das Lied, in dem es heißt, die Psychiatrie behandle nie die wirkliche Ursache seelischer Erkrankungen, Kapitalismuskritik transportiere: ''Ein Protestsong gegen den ausufernden Kapitalismus mit all seinen Scheinheiligkeiten; und so ist der Text gedacht und entstanden.''
\endscripture
