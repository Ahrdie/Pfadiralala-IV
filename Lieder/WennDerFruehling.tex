\beginsong{Wenn der Frühling kommt}[wuw={rumpel (Andreas Barth), BdP Löwenherz, Marburg}, jahr={1989}, siru={265}]

\beginverse
\endverse
\includegraphics[page=1]{Noten/WennDerFruehling.pdf}
\includegraphics[page=2]{Noten/WennDerFruehling.pdf}


% \beginverse\memorize
% Wenn der \[Em]Frühling \[Hm]kommt und die \[G]Vögel \[Em]ziehn
% und die \[C]düsteren Wolken nach \[D]Norden fliehn,
% wenn man \[Em]Freude und \[Hm]Freiheit \[G]dennoch ver\[Em]liert,
% weil der \[C]graue Alltag die \[D]Menschen regiert,
% \endverse


\beginverse
Wenn das \[Em]Feuer \[Hm]in der \[G]Kohte \[Em]schwelt,
manch \[C]einer von großer \[D]Fahrt erzählt,
wenn der \[Em]dampfende \[Hm]Tee übern \[G]Feuer \[Em]hängt,
und das \[C]Glück der Freiheit den \[D]Kummer verdrängt,
\endverse

% \beginchorus
% Dann ist es \[G]Zeit, die ledernen \[D]Hosen zu tragen,
% die \[C]alten verwaschenen \[H7]Klampfen zu schlagen und \[C]Aben\[D7]teuer zu be\[G]stehn.
% Und es \[G]lohnt sich die uralten \[D]Lieder zu singen,
% durch \[C]Wälder zu streifen und \[H7]Berge zu zwingen
% und die \[Em]uralte \[H7]Sonne \[Em]wiederzusehen, und die \[Em]uralte \[H7]Sonne zu \[Em]sehn.
% \endchorus
\printchorus

\beginverse
Und es ^kommt ein ^Frühling zu ^uns in das ^Land
und die ^Lieder, sie sind uns nicht ^unbekannt.
Lasst uns ^freudig ^in die ^Ferne ^ziehn
und gleich ^Vögeln dem Hier und ^Jetzt entfliehn.
\endverse

\printchorus

\endsong
