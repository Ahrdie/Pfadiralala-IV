\beginsong{The Green Fields of France}[wuw={Eric Bogle, 1976}, index={Well, how do you do}]

\beginverse\memorize
Well, \[G]how do you \[Em]do, young \[C]Willie Mc\[Am]Bride?
Do you \[D]mind, if I sit here down be\[G]side your \[C]grave\[G]side
and rest for a \[Em]while 'neath the \[C]warm summer \[Am]sun?
I've been \[D]walking all day, and \[G]I'm \[C]nearly \[G]done.
And I see by your \[Em]gravestone you \[C]were only nine\[Am]teen,
when you \[D]joined the great fallen in \[G]nineteen-\[D7]sixteen.
Well, I \[G]hope you died \[Em]quick and I \[Am]hope you died \[C]clean.
Or \[D]Willie McBride, was is it \[G]slow and \[D7]ob\[G]scene?
\endverse

\beginchorus
Did they \[D]beat the drum slowly, did they \[C]play the fife \[G]lowly,
did they \[D]sound the dead march, as they \[C]lowered you \[G]down,
did the \[C]band play the last post and \[D]chorus,
did the \[G]pipes play the ''\[C]Flowers of the \[D7]Fo\[G]rest''?
\endchorus

\beginverse
And did ^you leave a ^wife or a ^sweetheart be^hind?
In ^some loyal heart is your ^memory ^en^shrined.
And though you died ^back in ^nineteen-six^teen,
to ^that loyal heart you're for^ever ^nine^teen.
Or are you a ^stranger without ^even a ^name,
for^ever enshrined behind ^some old glass ^pane,
in an ^old photo^graph torn, ^tattered and ^stained
and ^faded to yellow in a ^brown lea^ther ^frame?
\endverse

\printchorus

\beginverse
The ^sun's shining ^down on these ^green fields of ^France,
the ^warm wind blows gently and the ^red pop^pies ^dance,
the trenches have ^vanished long ^under the ^plow.
No ^gas, no barbed wire, no guns ^fi^ring ^now!
But here in this ^graveyard it's ^still ''No Man's ^Land'',
the ^countless white crosses in ^mute witness ^stand
to ^man's blind in^difference to ^his fellow ^man
and a ^whole generation that were ^butchered ^and ^damned.
\endverse

% \printchorus

\beginverse
And I ^can't help but ^wonder, oh ^Willie Mc^Bride.
Do ^all those, who lie here, ^know why ^they ^died,
did you really be^lieve them, when they ^told you the ^cause,
did they ^really believe that this ^war would ^end ^wars?
Well, the suffering, the ^sorrow, the ^glory, the ^shame,
the ^killing and dying, it was ^all done in ^vain.
Oh ^Willie Mc^Bride, it all ^happened ^again
and a^gain, and again, and a^gain, and ^a^gain.
\endverse

% \printchorus

\endsong

\beginscripture{}
Das Lied beschreibt die Gedanken über ein junges Opfer des Ersten Weltkriegs in Flandern oder Nordfrankreich. Den Daten entsprechend könnte es sich um den in Authuille begrabenen William McBride handeln. Genauso gut könnte es jedoch auch sein, dass das Lied von einer fiktive Person handelt.
\endscripture
