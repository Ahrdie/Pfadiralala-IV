\beginsong{My Bonnie}[
    wuw={trad. Scottish},
]

\beginverse\memorize
My \[D]Bonnie lies \[G]over the \[D]ocean, my \[D]Bonnie lies \[E7]over the \[A]sea, \[A7]
my \[D]Bonnie lies \[G]over the \[D]ocean, oh \[G]bring back my \[A7]Bonnie to \[D]me.
\endverse

\beginchorus
\[D]Bring back, \[G]bring back, oh \[A]bring back, my \[A7]Bonnie to me, to \[D]me.
\[D]Bring back, \[G]bring back, oh \[A7]bring back my Bonnie to \[D]me.
\endchorus

\beginverse
Oh, ^blow ye winds ^over the ^ocean, oh ^blow ye winds ^over the ^sea, ^
oh, ^blow ye winds ^over the ^ocean, and ^bring back my ^Bonnie to ^me.
\endverse

\printchorus

\beginverse
Last ^night as I ^lay on my ^pillow, last ^night as I ^lay on my ^bed, ^
last ^night as I ^lay on my ^pillow, I ^dreamed that my ^Bonnie was ^dead.
\endverse

\printchorus

\beginverse
The ^winds have blown ^over the ^ocean, the ^winds have blown ^over the ^sea, ^
the ^winds have blown ^over the ^ocean, and ^brought back my ^Bonnie to ^me.
\endverse

\printchorus

\endsong

\beginscripture{}
1881 veröffentlichte Charles E. Pratt unter den Pseudonymen H. J. Fuller und J. T. Wood Noten für '' Bring Back My Bonnie to Me''. Theodore Raph schreibt in seinem 1964 erschienenen Buch American Song Treasury: 100 Favorites, dass das Lied in den 1870er Jahren in den Notenläden nachgefragt wurde und Pratt davon überzeugt werden konnte, eine Version des Liedes unter Pseudonymen zu veröffentlichen. Das Lied wurde ein großer Hit, besonders beliebt bei Gesangsgruppen an Colleges, aber auch bei allen anderen Gesangsgruppen.

Es exisiert eine pouläre Version der Gruppe ''Tony Sheridan and the Beatles'' von 1961, die als erste Single der Beatles betrachtet wird.
\endscripture
