\beginsong{Am Westermanns Lönstief}
    [wuw={trenk (Alo Hamm), Zugvogel Deutscher Fahrtenbund}, 
    siru={16}, 
    bo={22},
]

% \beginverse
% Am \[Am]Westermann Lönstief pfeift \[Dm]eisiger \[Am]Wind,
% uns \[Dm]schaukelt die \[Am]See wie die \[E]Mutter ihr \[Am]Kind.
% \[G]Am \[C]Westermanns \[G]Lönstief ist \[C]alles so \[G]grau,
% wir \[Am]fangen den \[E]Hering, den \[Am]Ka\[E]bel\[Am]jau
% \endverse
%
% \beginchorus
% Tschi\[Dm]ree macht die \[Am]See. Tschi\[E]ra, tschi\[Am]ree.
% Tschi\[Dm]ree macht die \[Am]See. Tschi\[E]ra-hahaha, tsch\[Am]iree.
% \endchorus

\beginverse
\endverse
\includegraphics[draft=false, page=1]{Noten/AmWestermanns.pdf}


\beginverse
Durch \[Am]Tage und Nächte wir \[Dm]kurven im \[Am]Nord, 
und \[Dm]hieven die \[Am]zappelnde \[E]Beute an \[Am]Bord.
\[G]Wir \[C]kehlen den \[G]Hering und \[C]salzen ihn \[G]ein, 
sind \[Am]voll unsere \[E]Kantjes, wir \[Am]fah\[E]ren \[Am]heim.
\endverse

\beginchorus
Tschi\[Dm]ree macht die \[Am]See. Tschi\[E]ra, tschi\[Am]ree.
Tschi\[Dm]ree macht die \[Am]See. Tschi\[E]ra-hahaha, tsch\[Am]iree.
\endchorus

\beginverse
Süd^wester, das Ölzeug und ^Isländer ^Wams, 
was ^nützen die ^Plünnen im ^Schneeflocken^tanz.
^Ein ^daumenbreit ^Schluck aus der ^Buddel mit ^Rum, 
das ^krempelt uns ^wieder 'ne ^Wei^le ^um.
\endverse

\printchorus

\beginverse
Springt ^über die Reling Jan ^Rasmus, Tschi^ree. 
Fass ^Taue, halt ^fest dich, sonst ^fährst du zur ^See.
^So ^mancher fuhr ^tief in den ^Meerkeller ^ein,
kommt ^nicht mehr ^heraus vor Sankt ^Nim^mer^lein.
\endverse

\repchorus{2}

\endsong

\beginscripture{}
\endscripture