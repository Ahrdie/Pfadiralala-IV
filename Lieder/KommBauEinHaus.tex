\beginsong{Komm' bau ein Haus}[
    wuw={Peter Janssens}, 
    jahr={1994},
    tonspur={88}, 
    gruen={40}, 
]

\beginverse
\endverse
\includegraphics[draft=false, width=1\textwidth]{Noten/KommBauEinHaus.pdf}	

\beginchorus
\[G]Komm', bau ein Haus, das \[C]uns beschützt,
\[G]pflanz' \[H7]einen \[Em]Baum, der \[A]Schatten wirft\[D]
und be\[C]schreibe den \[D]Himmel, der uns \[G]blüht\[H7] \[Em]
und be\[C]schreibe den \[D]Himmel, der uns \[G]blüht. \[D]
\endchorus

\beginverse\memorize
\[G]Lad' viele Kinder ein ins \[D]Haus, 
ver\[C]sammle sie\[D] bei unser'm \[G]Baum,
\[Em]lass sie dort fröhlich \[Am]tanzen,\[A7] wo \[D]keiner ihre \[A]Kreise \[D]stört,
\[C]lass sie dort\[D] lange \[G H7 Em]tanzen,\[C] wo der \[D]Himmel \[G]blüht. \[D]
\endverse

\printchorus

\beginverse
^Lad' viele Alte ein ins ^Haus, 
be^wirte sie^ bei unser'm ^Baum,
^lass sie dort frei er^zählen,^ von ^Kreisen die ihr ^Leben ^zog,
^lass sie dort^ frei er^zählen,^ wo der ^Himmel ^blüht. ^
\endverse

\printchorus

\beginverse
^Komm, wohn mit mir in diesem ^Haus, 
be^gieß' mit mir^ diesen ^Baum,
^dann wird die Freude ^wachsen,^ weil ^unser Leben ^Kreise ^zieht,
^dann wird die^ Freude ^wachsen,^ wo der ^Himmel ^blüht. ^
\endverse

\printchorus

\endsong
