\beginsong{Die Wochenschau}[wuw={Jannis Hansa, Nils Berkey, Jonas Höchst}, jahr={2011}, index={Ich sitz' auf dem Keramikring}]

\beginverse*
{\nolyrics Intro: \[Am] \[F] \[Am] \[F]}
\endverse

\beginverse\memorize
Ich \[Am]sitz auf dem Keramikring, und \[F]denk mir das ist ja ‘n‘ Ding,
\[Am]Griechenland ist Pleite. \[F]
Der \[Am]Euro ist fast nichts mehr Wert, in \[F]jedem Land ein Krisenherd,
ein \[Am]Hoch auf die alten \[F]Zeiten. \[G]
\endverse

\beginchorus
Bei \[C]allen den Kon\[G]flikten und Pro\[F]blemen auf der Welt,
\[C]wär‘ es doch mal \[G]Zeit, dass sich \[F]Besserung einstelt.
\[C]Doch wenn wir zu\[G]sammen steh’n und \[F]nicht nur auf der Stelle geh’n,
wirst du er\[C]kennen was die \[G]Welt im \[F]Innersten zu\[D]sammen hält. \[Am] \[F] \[Am] \[F]
\endchorus

\beginverse
Gad^dafis Herrschaft ist vorbei, dort ^unten sind die Menschen frei,
Ich ^bin gespannt, wie’s weitergeht...
^Öl im Golf von Mexico, ^Tod in Guantanamo,
ob ^sich die Welt noch lang so weiter^dreht? \[G]
\endverse

\beginchorus
\endchorus

\beginverse*
Die \[H]Sitzung ist vor\[G]bei,\[E] es wird Zeit \[D]aufzu\[C#]stehen.
Doch \[H]ein De\[G]tail\[E] hab ich noch \[D]über\[C#]seh’n.
Die \[H]Rolle neben \[G]mir\[E] ist nur noch \[D]Pappen\[C#]grau.
Hab‘ \[H]kein Pa\[G]pier \[E]neh’m ich halt die \[D]Wochen\[C#]schau!
Ich \[H]scheiße auf Gad\[G]dafi und \[E]all die andern \[D]Affen,\[C#]
die \[H]Peace \& Love und \[G]Weltfrieden \[E]einfach nicht \[C]raff\[H]en. 
\endverse

\beginchorus
Bei \[C]allen den Kon\[G]flikten und Pro\[F]blemen auf der Welt,
\[C]was ist schon mein \[G]Klopapier \[F]gegen all das Geld?
das \[C]ausgegeben  \[G]wird um Krieg und  \[F]Hass zu provozieren,
wenn  \[C]wir so weiter \[G]machen können  \[F]wir nur ver \[D]lier’n!\[Am] \[F] \[Am] \[F] \[Am]
\endchorus

\endsong
\beginscripture{}
Das Lied entstand auf den Hessischen Herbsttagen 2011 zum Thema ''Hippies'' in der Protestsong-AG.
\endscripture
