\beginsong{Jeden Morgen geht die Sonne auf}[
    txt={Hermann Claudius}, 
    txtjahr={1937}, 
    mel={Karl Marx}, 
    meljahr={1949}, 
    gruen={88}, 
    pfi={1}, 
    siru={150},
    buedel={181}, 
]

% \beginverse
% Jeden Morgen geht die Sonne auf
% in der Wälder wundersame Runde
% und die schöne, scheue Schöpferstunde
% jeden Morgen nimmt sie ihren Lauf.
% \endverse

\beginverse
\endverse
\includegraphics[draft=false, page=1]{Noten/JedenMorgenGehtDieSonneAuf.pdf}

\beginverse
\[A]Jeden \[E]Morgen aus den \[A]Wiesengründen heben weiße Schleier sich ins \[E]Licht.
Uns der \[A]Sonne \[D]Morgen\[E]gang zu \[A]künden, ehe sie das Wolken\[E]tor durch\[A]bricht.
\endverse

\beginverse
^Jeden ^Morgen durch des ^Waldes Hall, hebt der Hirsch sein mächtiges Ge^weih.
Der ^Pirol und ^dann die ^Vög'lein ^alle stimmen an die große ^Melo^dei.
\endverse

\endsong

\beginscripture{}
Trotz der namentlichen Ähnlichkeit sind der Komponist und der Philosoph Karl Marx zwei völlig verschiedene Personen.
\endscripture
