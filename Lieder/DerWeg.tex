\beginsong{Der Weg}[wuw={Herbert Grönemeyer}, jahr=2002, alb={Mensch}]

\beginverse\memorize
\[C]Ich kann nicht mehr seh'n\[Em], \[F]trau nicht mehr meinen \[C]Augen.
Kann kaum noch \[Em]glauben, \[F]Gefühle haben sich ge\[G]dreht.
Ich bin viel zu \[Em]träge,\[F] um aufzu\[C]geben.
Es wär' \[E]auch zu \[Am]früh, \[Fm] weil immer was \[C]geht.
\endverse

\beginverse
^Wir waren ver^schworen, ^wären füreinander ge^storben.
Haben den Regen ge^bogen, ^uns Vertrauen ge^lieh'n.
Wir haben ver^sucht, ^auf der Schussfahrt zu ^wenden.
^ Nichts war zu ^spät, ^aber vieles zu ^früh.
\endverse

\beginverse
^Wir haben uns ge^schoben, ^durch alle Ge^zeiten.
Haben uns ver^zettelt, ^uns verzweifelt ge^liebt.
Wir haben die ^Wahrheit,^ so gut es ging ver^logen.
Es war ein ^Stück vom ^Himmel, ^ dass es dich ^gibt.
\endverse

\beginchorus
Du hast jeden \[Em]Raum, \[F] mit Sonne ge\[C]flutet.
Hast jeden Ver\[Em]druss, ins \[F]Gegenteil ver\[G]kehrt.
Nordisch-\[C]nobel, deine \[F]sanftmütige \[Am]Güte. \[Em]
Dein unbändiger \[F]Stolz, \[G]das Leben ist nicht \[C]fair.
\endchorus

\beginverse
Den \[C]Film ge\[Em]tanzt, in einem \[F]silbernen Raum, \[C]
vom goldenen \[Em]Balkon, die Un\[F]endlichkeit be\[G]staunt.
Heillos ver\[C]sunken, \[G]trunken, und \[F]alles war \[Am]erlaubt. 
\[G]Zusammen \[F]im Zeit\[C]raffer, \[G]Mittsommernachts\[C]traum.
\endverse

\printchorus

\beginchorus
Dein \[C]sicherer \[Em]Gang, \[F]deine wahren Ge\[C]dichte,
deine heitere \[Em]Würde, dein uner\[F]schütterliches Ge\[G]schick.
Du hast der \[C]Fügung, \[F]deine Stirn ge\[Am]boten.\[Em]
Hast ihn nie ver\[F]raten, \[G]deinen Plan vom \[C]Glück.
\[G]Deinen Plan vom \[C]Glück.
\endchorus

\beginverse
\[C]Ich gehe nicht \[Em]weg, \[F]hab' meine Frist ver\[C]längert.
Neue Zeit\[Em]reise,\[F] offene \[G]Welt.
Habe dich \[Em]sicher, in\[F] meiner \[C]Seele.
Ich \[E]trag' dich \[Am]bei mir, \[Fm] bis der Vorhang \[C]fällt.
Ich \[E]trag' dich \[Am]bei mir, \[Fm] bis der Vorhang \[C]fällt.
\endverse

\endsong
\beginscripture{}
Wenige Monate nach dem Erscheinen von Grönemeyers zehntem Album ''Bleibt alles anders'' starben innerhalb weniger Tage zunächst einer seiner Brüder und dann seine Frau an Krebs. Grönemeyer zog sich nach diesem doppelten Schicksalsschlag aus der Öffentlichkeit zurück und fühlte sich lange nicht in der Lage, Musik zu machen. Etwa ein Jahr später konnte er sich zum Weitermachen aufraffen und erstellte in 14 monatiger, fast therapeutischer Arbeit das Album ''Mensch'', in dem er sein erlebtes Schicksal verarbeitete. In ''Der Weg'' richtet er sich an seine verstorbene Frau.
\endscripture

\begin{intersong}

\end{intersong}
