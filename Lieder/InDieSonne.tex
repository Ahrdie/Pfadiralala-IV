\beginsong{In die Sonne, die Ferne, hinaus}[
    wuw={Wilhelm Sell}, 
    txt={Kurt Hoppstädter, Nerother Wandervogel (4. Strophe)},
    bo={208},
]

\beginverse
% In die \[E]Sonne, die \[H7]Ferne hin\[E]aus, lasst die Sorgen, den \[H7]Alltag zu \[E]Haus.
% \lrep Von \[A]Bergen über grüne Auen \[E]lohnt es sich zu schauen \[H7]in die weite \[E]Welt. \rrep
\endverse
\includegraphics[draft=false, width=1\textwidth]{Noten/InDieSonne.pdf}

\beginverse\memorize
Kommt der \[E]Frühling zu \[H7]uns in das \[E]Land, 
nimm das Ränzel, die \[H7]Klampfe zur \[E]Hand.
\lrep Durch \[A]fremde Lande wollen fahren 
\[E]junge, frohe Scharen \[H7]in die weite \[E]Welt. \rrep
\endverse

\beginverse
Wenn das ^Feuer die ^Nacht weit er^hellt, 
und wir stehen zu^sammenge^sellt,
\lrep dann ^klingen uns're alten Lieder 
^von den Bergen wider ^in die weite ^Welt. \rrep
\endverse

\beginverse
Wenn auch ^Tod und Ver^derben uns ^droh'n, 
wir hoffen, wir ^kommen da^von,
\lrep denn wir ^lieben das Dasein auf der Erden, 
^immer neues Werden ^in der weiten ^Welt. \rrep
\endverse

\endsong
