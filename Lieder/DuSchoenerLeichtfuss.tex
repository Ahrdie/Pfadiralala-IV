\beginsong{Du schöner Leichtfuß}[wuw={venija (Eva Sophia Kuhn), Laninger Wandervogel}, biest={544}]

% \beginchorus
% Du schöner \[Am]Leichtfuß, mit diesen Augen,
% ich \[F]würde dich so gern ganz bald mal \[C]wiederseh'n.
% Mich \[Dm]selbst nicht mehr versteh'n denn dich \[Am]brauch ich,
% um den \[F]Sommer anzuhalten, da\[C]mit nicht so bald wieder \[G]Winter wird,
% kalter \[C]Winter wird, einsamer \[E]Winter wird.
% \endchorus

% \beginverse
% Mit dir wär' \[F]jeden Tag Markt \[C]unter meinem Fenster,
% \[Dm]du wärst der, über den ich \[Am]schimpfen kann,
% mit dem Ge\[G]müsemann oder der \[E]Blumenfrau.
% Und viel\[F]leicht hängt Sie dann mal einen \[C]Mistelzweig über \[E]uns?
% \endverse

\beginverse
\endverse
\includegraphics[page=1]{Noten/DuSchoenerLeichtfuss.pdf}

\beginchorus
Du schöner \[Am]Leichtfuß, mit diesen Augen,
ich \[F]würde dich so gern ganz bald mal \[C]wiederseh'n.
Mich \[Dm]selbst nicht mehr versteh'n denn dich \[Am]brauch ich,
um den \[F]Sommer anzuhalten, da\[C]mit nicht so bald wieder \[G]Winter wird,
kalter \[C]Winter wird, einsamer \[E]Winter wird.
\endchorus

\beginverse\memorize
Das, \[F]was in mir ist, es würde \[C]nie wieder leise, 
beim \[Dm]Bier auf’m Platz wär' ich \[Am]nicht mehr allein.
Wir \[F]zwei würden uns \[C]nie verraten 
und ich \[G]könnte jede Nacht ganz re\[E]al bei dir sein.
\endverse

\printchorus

\beginverse
Im ^alten Sommerregen würden wir ^tanzen im Hof, 
der ^Biergarten könnte das ^Hauptquartier sein?
Ich \[F]glaub fast, für dich find ich \[C]nie die besten Worte, 
doch ich \[G]weiß es könnte \[E]großartig sein.
\endverse

\printchorus

\endsong
