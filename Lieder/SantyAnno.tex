\beginsong{Santy Anno}[
    wuw={Traditionell, J. M. Hunt}, 
    jahr={1935}, 
    gruen={240},
    pfii={217},
    siru={275}, 
]

\beginverse\memorize
We're \[Em]sailing down the river from \[D]Liverpool, 
heave a\[Em]way, Santy \[D]Anno.
A\[Am]round Cape Horn to \[D]Frisco Bay, 
all \[Em]on the \[D]plains of \[Em]Mexico.
\endverse

\beginchorus
So \[Em]heave her up and a\[D]way we'll go, heave a\[Em]way, Santy \[D]Anno.
\[Am]Heave her up and a\[D]way we'll go, all \[Em]on the \[D]plains of \[Em]Mexico.
\endchorus

\beginverse
She's a ^fast clipper ship and a ^bully good crew,
heave a^way, Santy ^Anno.
A ^down-East Yankee for her ^captain, too,
all ^on the ^plains of ^Mexico.
\endverse

\beginverse
There's ^plenty of gold, so ^I've been told,
heave a^way, Santy ^Anno.
There's ^plenty of gold, so ^I've been told,
way out-^West to ^Cali^fornio.
\endverse

\beginverse
^Back in the days of ^Forty-nine,
heave a^way, Santy ^Anno.
Those ^are the days of the ^good old times,
all ^on the ^plains of ^Mexico.
\endverse

\beginverse
When ^Zacharias Taylor ^gained the day,
heave a^way, Santy ^Anno.
He ^made poor Santy ^run away,
all ^on the ^plains of ^Mexico.
\endverse

\beginverse
^General Scott and ^Taylor, too,
heave a^way, Santy ^Anno.
Made ^poor Santy meet his ^Waterloo,
all ^on the ^plains of ^Mexico.
\endverse

\beginverse
When ^I leave ship, I'll ^settle down,
heave a^way, Santy ^Anno.
And ^marry a girl named ^Sally Brown,
all ^on the ^plains of ^Mexico.
\endverse

\beginverse
Santy ^Anno was a ^good old man,
heave ^away, Santy ^Anno.
'Til he ^got into war with your ^Uncle Sam,
all ^on the ^plains of ^Mexico.
\endverse

\endsong

\beginscripture{}
Traditionell, nach J. M. Hunt (''Sailor Dad''), aus Marion, Virginia (1935). Erfasst, angepasst und arrangiert von John A. and Alan Lomax in ''Our Singing Country''.

Das Lied wurde unter anderem von ''The Highwaymen'' populär gemacht. Die Strophen 2, 5, 6 und 8 wurden von ihnen fortgelassen und sind in einigen anderen Liederbüchern deshalb nicht abgedruckt.
\endscripture