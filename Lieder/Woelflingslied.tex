\beginsong{Wölflingslied}[index={Komm lauf' mit uns hinaus}]

% \beginverse
% Kommt lauf mit uns hinaus.
% Wir bleiben nicht zuhaus'
% Komm Wöflinge geht auf Jagd.
% \endverse
%
% \beginverse
% Im Walde kennen wir jede Pflanze jedes Tier,
% hören den leisesten Tritt,
% ob trocken oder nass,
% es macht uns großen Spaß.
% Komm mit kleiner Wolf
% kommt mit!
% \endverse

\beginverse
\endverse
\includegraphics[page=1]{Noten/Woelflingslied.pdf}

\beginverse
Und \[G]sind wir dann im Heim, wir \[C]können nicht ruhig sein, 
das \[D7]ist nicht unsere \[G]Art.
\lrep Wir \[G]singen froh ein Lied, ein' \[Am]jeden reißt es mit,
\[D7]machen den größten Krach.
Und \[G]ist die Runde aus, dann \[C]zieh'n wir froh nach Haus.
Mach \[D7]mit, kleiner Wolf, mach \[G]mit! \rrep
\endverse

\endsong

\beginscripture{}
Dieses Lied wird in manchen Stämmen im Abschlusskreis gesungen. Meist nur die erste zwei Strophen. Danach fragt der Akela: ''Wölflinge wollt ihr euer bestes tun?'', worauf hin die Wölflinge mit ''Ja, wir wollen unser bestes tun!'' antworten und den Abschlusskreis mit ''Gut Jagd!'' und dem Wölflingszeichen beenden.
\endscripture
