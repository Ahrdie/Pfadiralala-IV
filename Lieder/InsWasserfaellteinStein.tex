\beginsong{Ins Wasser fällt ein Stein}[
    wuw={Manfred Siebald, Kurt Kaiser}, 
    pfiii={66}, 
    gruen={41}, 
    kssiv={318}, 
    siru={136},
]

\beginverse
Ins \[D]Wasser fällt ein \[F#m]Stein, ganz \[G]heimlich, still und \[A]leise;
und \[D]ist er noch so \[F#m]klein, er \[G]zieht doch weite \[A]Kreise.
Wo \[G]Gottes große \[D]Liebe \[G]in einen \[D]Menschen \[Hm]fällt,
da \[G]wirkt sie \[D]fort in \[G]Tat und \[D]Wort, hi\[Em]naus in \[A]uns're \[D]Welt.
\endverse

\beginverse
Ein \[D]Funke, kaum zu \[F#m]seh'n, ent\[G]facht doch helle \[A]Flammen.
Und \[D]die im Dunkeln \[F#m]steh'n, die \[G]ruft der Schein zu\[A]sammen.
Wo \[G]Gottes große \[D]Liebe \[G]in einem \[D]Menschen \[Hm]brennt,
da \[G]wird die \[D]Welt vom \[G]Licht er\[D]hellt, da \[Em]bleibt nichts, \[A]was uns \[D]trennt.
\endverse

\beginverse
Nimm ^Gottes Liebe ^an, du ^brauchst dich nicht allein ^zu müh'n,
denn ^seine Liebe ^kann in ^deinem Leben ^Kreise zieh'n.
Und ^füllt sie erst dein ^Leben ^und setzt sie ^dich in ^Brand,
gehst ^du hi^naus, teilst ^Liebe ^aus, denn ^Gott füllt ^die die ^Hand.
\endverse

\endsong
