\beginsong{Unter den Toren}[
    wuw={olka (Erich Scholz), mac (Erik Martin)}, 
    mel={Little Pink},
    meljahr={2020},
    pfii={29}, 
    pfiii={14}, 
    bo={338}, 
    gruen={162}, 
    kssiv={12}, 
    siru={238},
    tonspur={542}, 
]

\beginverse
\[Emmaj7]Unter den Toren im \[Dm7]Schatten der Stadt schläft man \[Cmaj7]gut, wenn man sonst keine \[H7]Herberge hat.
\[Cmaj7]Keiner der fragt nach \[H7]woher und wohin und zu \[Cmaj7]kalt ist die \[H7]Nacht für Gen\[Esus4]darmen.
\endverse

\beginchorus
\[Gm]He\[Dm]-jo, ein \[Gm]Feuerlein \[Dm]brennt, \[Em]kalt ist es \[Esus4]für Gen\[H/hb6]darmen.
\[Gm]He\[Dm]-jo, ein \[Gm]Feuerlein \[Dm]brennt, \[Em]kalt ist es \[A]für Gen\[H]darmen.
\endchorus

\beginverse
^Silberne Löffel und ^Ketten im Sack legst du ^besser beim Schlafen dir
^unters Genack. ^Zeig' nichts und sag nichts, die ^Messer sind stumm
und zu ^kalt ist die ^Nacht für Gen^darmen.
\endverse

\beginverse
^Greif' nach der Flasche, doch ^trink' nicht zu viel deine ^Würfel sind gut,
aber ^falsch ist das Spiel. ^Spuck in die Asche und ^schau' lieber zu 
denn zu ^kalt ist die ^Nacht für Gen^darmen.
\endverse

\beginverse
^Rückt dir die freundliche ^Schwester zu nah, das ist ^gut für die
Wärme mal ^hier und mal da. ^Keiner im Dunkeln ver^rät sein Gesicht
und zu ^kalt ist die ^Nacht für Gen^darmen.
\endverse

\beginverse
^Geh' mit der Nacht, eh der ^Frühnebel steigt, nur das ^Feuer bleibt stumm
und das ^Scheit, das verschweigt. ^Lass' nichts zurück und ver^giss', was du 
sahst, denn die ^Sonne bringt ^bald die Gen^darmen.
\endverse

\endsong
