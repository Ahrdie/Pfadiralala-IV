\beginsong{Soon May the Wellerman Come}[
	wuw={F. R. Woods, John Smith A. B., D. H. Rogers},
	jahr={circa 1860},
]

\beginverse
There \[Am]once was a ship that put to sea, 
The \[Dm]name of the ship was the \[Am]Billy of Tea 
The winds blew up, her bow dipped down, 
O \[E]blow, my bully boys, \[Am]blow.
\endverse

\beginchorus
\[F]Soon may the \[C]Wellerman come 
And \[Dm]bring us sugar and \[Am]tea and rum. 
\[F]One day, when the \[C]tonguin' is done, 
We'll \[E]take our leave and \[Am]go.
\endchorus


\beginverse
She ^had not been two weeks from shore 
When ^down on her a ^right whale bore. 
The captain called all hands and swore 
He'd ^take that whale in ^tow. 
\endverse


\beginverse
Be^fore the boat had hit the water 
The ^whale's tail came ^up and caught her. 
All hands to the side, harpooned and fought her 
^When she dived down be^low. 
\endverse

\beginverse
No ^line was cut, no whale was freed; 
The ^Captain's mind was ^not of greed, 
But he belonged to the whaleman's creed; 
She ^took the ship in ^tow. 
\endverse

\beginverse
For ^forty days, or even more, 
The ^line went slack, then ^tight once more. 
All boats were lost (there were only four) 
But ^still the whale did ^go. 
\endverse

\beginverse
As ^far as I've heard, the fight's still on; 
The ^line's not cut and the ^whale's not gone. 
The Wellerman makes his regular call 
To the ^Captain, crew, and ^all.
\endverse

\endsong

\beginscripture{}
Ab 1833 lieferten die "{}Wellermen"{} auf Schiffen der Weller Brothers aus Sydney von ihrem Stützpunkt in Otakou aus Proviant an die neuseeländischen Walfangstationen an der Küste. Das Lied besingt die Veränderungen über die Jahre ihrer Aufgabe und des Walfangs. Das Lied wurde Anfang 2021 über die Social Media Platform TikTok in duzenden Versionen gecovert und gelangte zu großer Bekanntheit.
\endscripture