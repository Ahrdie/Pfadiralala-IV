\beginsong{Café Oriental}[
    wuw={Vico Torriani}, 
    jahr={1961},
    index={Im Orient gibt's ein Lokal},
    biest={576},
    ]

\beginverse\memorize
Im Ori\[Am]ent gibt's ein Lokal, \echo{das Café Orien\[E]tal,}
jeder Scheich war schon einmal, \echo{im Café Orien\[Am]tal.}
Dies' Lokal ist ein Magnet dort gibt's Frauen ohne \[Dm]Zahl
und wer so was sucht, der \[Am]geht \echo{ins Ca\[E]fé Orien\[Am]tal.}
\[Am]Lei-lei lei-lei \[E]lei-la-lei, lei-lei lei-la-lei,
\[Am]Lei-lei lei-lei \[E]lei-la-lei, lei-lei lei-la-\[Am]la. Orien\[A#m]tal!
\endverse

\beginverse
Eine \[A#m]war besonders schön \echo{im Café Orien\[F]tal}
sie sah aus wie die Loren \echo{im Café Orien\[A#m]tal}
Herrlich war ihr Dekolleté, sie war schlank und so \[D#m]schmal
und war braun wie der Kaf\[A#m]fee \echo{im Ca\[F]fé Orien\[A#m]tal}
\[A#m]Lei-lei lei-lei \[F]lei-la-lei, lei-lei lei-la-lei,
\[A#m]Lei-lei lei-lei \[F]lei-la-lei, lei-lei lei-la-\[A#m]la. Orien\[Hm]tal!
\endverse

\transpose{+2}
\beginverse
Ich ging ^lächelnd auf sie zu \echo{im Café Orien^tal,}
bat sie um ein Rendezvous \echo{im Café Orien^tal.}
Kaum war ich in ihrer Näh', flog ich raus aus dem ^Saal
denn ihr Mann war der Por^tier \echo{vom Ca^fé Orien^tal!}
^Lei-lei lei-lei ^lei-la-lei, lei-lei lei-la-lei,
^Lei-lei lei-lei ^lei-la-lei, lei-lei lei-la-^la. Orien^tal!
\endverse

\transpose{+1}
\beginverse
Doch weil ^ich so gerne bin \echo{im Café Orien^tal,}
geh ich morgen wieder hin \echo{ins Café Orien^tal.}
Aber lacht mich eine an, frag ich erst sie ein^mal:
''Sag'n Sie, hab'n Sie einen ^Mann \echo{im Ca^fé Orien^tal}?''
^Nei-nei nei-nei ^nei-na-nei, nei-nei nei-na-nein,
^Nei-nei nei-nei ^nei-na-nei, nei-nei nei-na-^na: \echo{Im Ca\[E]fé Orien\[Am]tal!}
\endverse


\endsong
