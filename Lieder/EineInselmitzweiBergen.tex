\beginsong{Eine Insel mit zwei Bergen}[wuw={Augsburger Puppenkiste}, pfiii={195}]

\markboth{\songtitle}{\songtitle}

\beginverse
\[D]Ei\[A]ne \[D]Insel mit zwei Bergen und im \[A]tiefen, weiten Meer.
Mit viel \[A7]Tunnels und Geleisen und dem \[D]Eisenbahnverkehr.
Nun, wie mag die Insel heißen, ringshe\[A]rum ist schöner Strand.
Jeder \[A7]sollte einmal reisen, In das \[D]schöne Lummerland!
\endverse

\beginverse*
{\nolyrics Zwischenspiel: \[Em] \[D] \[A] \[D] \rep{2}}
\endverse

\beginverse
^Ei^ne ^Insel mit zwei Bergen und dem ^Fotoatelier.
In dem ^letzten macht man Bilder, auf den ^ersten 'dulijö'.
Diese Breiten, diese Tiefen, diese ^Höhen sind bekannt. 
Und man ^spricht von den Motiven auf dem ^schönen Lummerland!
\endverse

\beginverse*
{\nolyrics Zwischenspiel: \[Em] \[D] \[A] \[D] \rep{2}}
\endverse

\beginverse
^Ei^ne ^Insel mit zwei Bergen und dem ^Fernsprechtelefon,
wählt man ^nur die richtige Nummer, klappt auch ^die Verbindung schon.
'Hallo, hier ist falsch verbunden!' 'Wollen ^sie sich jetzt beschwer'n?'
'Nein, wa^rum? Das kann passier'n!' 'Also ^dann auf Wiederhör'n!'
\endverse

\beginverse*
{\nolyrics Zwischenspiel: \[Em] \[D] \[A] \[D] \rep{2}}
\endverse

\beginverse
^Ei^ne ^Insel mit zwei Bergen und dem ^Laden von Frau Waas:
Husten^bonbons, Alleskleber, Regen^schirme, Leberkas,
Körbe, Hüte, Lampen, Würste, Blumen^kohl und Fensterglas,
Leder^hosen, Kukucksuhren und noch ^dies und dann noch das!
\endverse

\beginverse*
{\nolyrics Zwischenspiel: \[Em] \[D] \[A] \[D] \rep{2}}

\endverse


\endsong

\beginscripture

\endscripture

\begin{intersong}

\end{intersong}