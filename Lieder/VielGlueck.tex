\beginsong{Viel Glück}[
    txt={Hans-Ulrich Treichel}, 
    txtjahr={1986},
    mel={plauder (Jörg Seyffarth), Rudi Rumstajn},
    meljahr={2016\\},
    lager={Überbündisches Treffen \emph{''ÜT''} 2017},
]


% \beginverse\memorize
% Viel \[Hm]Glück und \[F#]liebe \[Hm]Grüße und \[A]hunderttausend \[D]Küsse.
% An \[H7]alle Hunger\[Em]leider, an \[G]alle \[A]Beutel\[D]schneider,
% an \[H7]alle Mauer\[Em]springer, an \[F#]alle Fahnen\[G]schwinger.
% \endverse
%
% \beginchorus
% La \[Em]lei-la-lei-lala --- lala \[Hm]lei-la-lei-lala,
% lala \[Em]lei-la-lei-la\[F#]la --- lala \[Hm]lei-la-lei-lala.
% \endchorus

\beginverse
\endverse 
\centering\includegraphics[width=1\textwidth]{Noten/VielGlueck.pdf}

\beginverse
Viel \[Hm]Glück und \[F#]wunde \[Hm]Füße und \[A]hunderttausend \[D]Bisse,
An \[H7]alle, die wir \[Em]kennen, an \[G]alle, \[A]die noch \[D]rennen, 
Viel \[H7]Glück und lass dir's \[Em]gut gehen, und \[F#]lass Dir einen \[G]Hut steh'n.
\endverse

\beginchorus
La \[Em]lei-la-lei-lala --- lala \[Hm]lei-la-lei-lala,
lala \[Em]lei-la-lei-la\[F#]la --- lala \[Hm]lei-la-lei-lala.
\endchorus


\beginverse
Und ^lass Dir ^einen B^art steh'n, einen ^feinen Ziegen^bart steh'n,
auf ^das wir nie ver^gessen, die ^Liebe ^und das ^Fressen,
auf ^das sich bald der ^Wind dreht, so ^gut wie's halt ^eben geht.
\endverse

\printchorus

\beginverse
Viel ^Glück und ^bess're ^Zeiten, beim ^Laufen und beim ^Reiten
und ^auch beim Diskus^werfen, wir ^woll'n die ^Messer ^schärfen,
und ^leg den Strick jetzt ^gut weg, denn ^du, du kennst ja ^das Versteck.
\endverse

\repchorus{2}

\endsong

\beginscripture{}
% Das Lied war Lagerlied des Überbündischen Treffens 2017 auf dem Allenspacher Hof auf der schwäbischen Alb.
Der Text verkörpert ein teilweise trotziges Auflehnen gegen bekannte Welten und Ordnungen, gepaart mit einer gewissen jugendlichen Melancholie. Der Text wird in einer veränderten Fassung gesungen, die plauder (Zugvogel) umgeschrieben hat, um das Gedicht besser singbar zu machen.
\endscripture

