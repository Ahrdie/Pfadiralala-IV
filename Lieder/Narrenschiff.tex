\beginsong{Das Narrenschiff}[wuw={Reinhard Mey}, jahr={1998}, alb={Flaschenpost}, index={Das Quecksilber fällt}]

\interlude{Anfang: \[Em C Am D] \rep{2} }

\beginverse\memorize
Das \[Em]Quecksilber fällt, die Zeichen stehen auf Sturm,
Nur blödes \[C]Kichern und Keifen vom Kommandoturm
Und ein \[Am]dumpfes Mahlen \[D]grollt aus der \[Em]Maschine.
Und \[Em]rollen und Stampfen und schwere See,
Die Bordka\[C]pelle spielt "Humbatäterä"
Und ein \[Am]irres Lachen \[D]dringt aus der La\[G]trine.
Die \[Am]Ladung ist faul, die Papiere fingiert,
Die \[Em]Lenzpumpen leck und die Schotten blockiert,
Die \[Am]Luken weit offen und \[C]alle Alarmglocken \[H7]läuten.
Die Seen \[C]schlagen mannshoch in den Laderaum
Und\[Am] Elmsfeuer züngeln vom Ladebaum,
Doch \[Am]keiner an \[G]Bord ver\[F#]mag die \[Em]Zeichen zu \[H7]deuten!
\endverse

\beginchorus
Der \[Em]Steuermann lügt, der Kapitän ist betrunken
und der \[D]Maschinist in dumpfe Lethargie versunken,
die \[C]Mannschaft lauter meineidige Halunken,
der \[H7]Funker zu feig' um SOS zu funken.
Kla\[C]bautermann führt das \[Am]Narrenschiff,
volle \[C]Fahrt voraus\[D] und \[Hm]Kurs auf's \[Em]Riff.
\endchorus

\interlude{Zwischenspiel: \[Em C Am D] \rep{2} }

\beginverse
Am Hori^zont wetterleuchten die Zeichen der Zeit:
^Niedertracht und Raffsucht und Eitelkeit.
Auf der ^Brücke tummeln sich ^Tölpel und Einfal^tspinsel.
Im ^Trüben fischt der scharfgezahnte Hai,
Bringt seinen ^Fang ins Trockne, an der Steuer vorbei,
Auf die ^Sandbank, bei der ^wohlbekannten Scha^tzinsel.
Die andern ^Geldwäscher und Zuhälter, die warten schon,
Bor^dellkönig, Spielautomatenbaron,
Im ^hellen Licht, niemand ^muss sich im Dunkeln rum^drücken
In der Ba^nanenrepublik, wo selbst der Präsident
Die ^Scham verloren hat und keine Skrupel kennt,
Sich mit dem ^Steuer^dieb im Ge^folge ^zu ^schmücken.
\endverse

\printchorus

\beginverse
Man hat sich ^glatt gemacht, man hat sich arrangiert,
all die ^hohen Ideale sind havariert.
Und der ^große Rebell, der nicht ^müd' wurde zu ^streiten,
mutiert zu ^einem servilen, gift'gen Gnom.
Und singt ^lammfromm vor dem schlimmen alten Mann in Rom
seine ^Lieder, fürwa^hr: Es ändern sich die ^Zeiten!\newpage
Einst junge ^Wilde sind gefügig, fromm und zahm,
ge^kauft, narkotisiert und flügellahm,
Tauschen ^Samtpfötchen für die ^einst so scharfen ^Klauen.
Und eitle ^Greise präsentier'n sich keck
mit immer ^viel zu jungen Frauen auf dem Oberdeck,
die ihre ^schlaffen ^Glieder ^wärmen und i^hnen das ^Essen vorkauen.
\endverse

\printchorus

\beginverse
Sie rüsten ^gegen den Feind, doch der Feind ist längst hier.
Er hat die ^Hand an deiner Gurgel, er steht hinter dir.
Im ^Schutz der Paragraphen mischt er ^die gezinkten ^Karten.
Jeder ^kann es sehen, aber alle sehen weg,
Und der ^Dunkelmann kommt aus seinem Versteck
Und ^dealt unter aller Augen ^vor dem Kinder^garten.
Der ^Ausguck ruft vom höchsten Mast: Endzeit in Sicht!
Doch sie sind ^wie versteinert und sie hören ihn nicht.
Sie ^ziehen wie Lemminge in ^willenlosen ^Horden.
Es ist, als ^hätten alle den Verstand verlor'n,
Sich zum ^Niedergang und zum Verfall verschwor'n,
Und ein ^Irrlicht ^ist ihr ^Leuchtfeu^er ge^worden.
\endverse

\printchorus

\endsong