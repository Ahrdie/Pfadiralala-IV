\beginsong{Im Kreisverkehr}[
    wuw={Herzdamen aus dem Zugvogel},
    % (Nora Lillich, Anne-Gesche Lipp, Marie Noda, Ronja Scholz) 
    jahr={2013},
    ]

% \beginverse\memorize
% \[Hm]Stundenlang, \[A]tagelang, \[G]nächtelang, \[F#]wochenlang,
% \[Hm]manchmal auch \[A]öfter und \[G]mehr,
% \[C]hält uns der \[H]Alltag auf \[Em]Trap - fest im \[C]Gang,
% drückt uns im \[H]Kopf drückt uns \[Em]schwer,
% dreht sich \[C]weiter im \[G]Kreis,
% \endverse
%
% \beginverse
% Im ^Kreise, die ^Erde sich ^einmal am ^Tag,
% Stunden, ^blühen, ver^welken, ver^weh'n.
% ^Keiner er^innert, was ^vor ihm mal ^lag,
% jeder hat ^Angst bald zu ^seh'n,
% Angst vom ^weiter seh'n, \[A]wei\[Hm]ter \[Em]geh'n.
% \endverse

\beginverse
\endverse
\centering\includegraphics[width=1\textwidth, page=1]{Noten/ImKreisverkehr.pdf}
\centering\includegraphics[width=1\textwidth, page=2]{Noten/ImKreisverkehr.pdf}

\beginverse
\endverse

\beginverse\memorize
\[Hm]Halten bei \[A]Rot, es wird \[G]Gelb schließlich \[F#]Grün, 
doch du \[Hm]passt den Mo\[A]ment gar nicht \[G]ab.
\[C]Dann bei Kirsch\[H]grün sich um \[Em]Tempo be\[C]müh'n, 
links mal im \[H]Kreisverkehr \[Em]rum. Und dann \[C]weiter im \[G]Fluss,
\endverse

\beginverse
Im ^Fluss, trink' die ^Flasche, da^mit sie mich ^trägt, 
denn be^drohlich schwappt ^Trübsal ins ^Boot.
^Halte mich ^fest, schau zu^rück, mich er^schlägt
die Macht dieser ^großen See^not einer ^Boje im \[A]Park\[Hm]ver\[Em]bot.
\endverse

\beginchorus
\[Em]Frei bleibt, \[Am]wer 'nen Gang zurück schraubt, \[Hm]wen der Fahrtwind \[C]trägt. \[D]
\[Em]Dreh dich \[C]um, was überseh'n, was manch \[Am]Zei\[Hm]chen am \[Am]Wegrand \[F#]ver\[Hm]rät. 
\endchorus

\interlude{Zwischenspiel: \[Hm] ~~ \[C H Em C] ~~ \[C H Em Em] ~~ \[C G-Hm Em]}

% \beginverse
% ^Stundenlang, ^tagelang, ^nächtelang, ^wochenlang,
% ^manchmal auch ^öfter und ^mehr,
% ^hält uns der ^Alltag auf ^Trap - fest im ^Gang,
% drückt uns im ^Kopf drückt uns ^schwer,
% dreht sich ^weiter im \[A]Kreis\[Hm]ver\[Em]kehr.
% \endverse

\beginverse
wie 1.
\endverse

\endsong
