\beginsong{Wilde Reiter}[
    wuw={trenk (Alo Hamm), Zugvogel Deutscher Fahrtenbund}, 
    jahr={1961},
    ]

% \beginverse\memorize
% \[D]Wilde Reiter, \[A]immer weiter \[Hm]auf der hohen \[F#]Straßenleiter
% \[G]jagen wir durch \[D]Tag und Nacht zum \[A]Meere.
% \[D]In Erinner\[A]ung der Zeiten, \[Hm]da durch Buchten \[F#]stolze Weiten
% \[G]uns befreit von \[D]aller Erden \[A]schwe\[D]re.
% \endverse

\beginverse
\endverse
\includegraphics[page=1]{Noten/WildeReiter.pdf}


\beginverse\memorize
\[D]Blumengärten, \[A]Ährenfelder, \[Hm]hohe Kronen \[F#]starker Wälder,
\[G]immer neuen \[D]Lebens bunter \[A]Fülle.
\[D]Alte Trachten, \[A]gelbe Dünen, \[Hm]schwarzgelackte \[F#]Holzpantinen,
\[G]Mädchenlachen \[D]und der Dämm'rung \[A]Stil\[D]le.
\endverse

\beginchorus
\lrep Je pense à \[D]vous, Madmoiselle, je pense à \[G]vous, Madmoiselle, 
à la \[D]terre et à la \[A]mer, Cap Frè\[D]hel. \rrep
\endchorus

\beginverse
^Die bretoni^sche Kapelle - ^Dudelsack und ^die Gesänge,
^Bombardon- im ^Violett der ^Heide.
^Brosche, Spitze, ^Silberspange - ^unter'm Himmel ^der Bretangne,
^geh'n die Leute ^in Brokat und ^Sei^de.
\endverse

\printchorus

\beginverse
^Diesem Land sind ^wir verschrieben, ^Land und Leute ^muss man lieben,
^seine Dörfer, ^Brunnen und Ta^vernen.
^Immer zieht noch ^durch Gedanken ^Glück, das wir am ^Wege fanden
^abends unter ^Sternen und La^ter^nen.
\endverse

\printchorus

\endsong

\begin{intersong}
\ifthenelse{\boolean{pics}}{
    \ThisLRCornerWallPaper{1}{Bilder/Lagerfeuer}
}{}
\end{intersong}