\beginsong{Stillleben}[wuw={rumpel (Andreas Barth), BdP Stamm Löwenherz, Marburg}, siru={58}, biest={540}, index={Das ist der Morgen nach langen Nächten}]

% \beginverse\memorize
% Dort auf dem \[Am]Tisch noch ein paar leere \[Dm]Flaschen,
% ein dicker \[G]Mönch lacht stolz vom Eti\[Am]kett.
% Ein Kartenspiel noch kann mein Blick er\[Dm]haschen
% und frische \[E]Kerben auf dem dunk'len Brett.
% \endverse
%
% \beginchorus
% \lrep Das ist der \[Am]Morgen nach langen \[Dm]Nächten, das ist das \[G]Ende wohl'ger Dunkel\[C]heit,
% die uns die \[Am]Sorgen die uns eng um\[Dm]flechten, vergessen \[E]läst und sonderbar be\[(Am)]freit. \rrep
% \endchorus
\beginverse
\endverse
\includegraphics[page=1]{Noten/Stillleben.pdf}


\beginverse
Das Spiel ist \[Am]aus, die Kerzen sind ver\[Dm]flossen,
verraucht der \[G]Docht, verglimmt im letzten \[Am]Loh’n.
Die letzten Schlucke sind schon lang \[Dm]genossen.
Leer ist der \[E]Kruge Glaskristall und Ton.
\endverse

\beginchorus
\lrep Das ist der \[Am]Morgen nach langen \[Dm]Nächten, 
das ist das \[G]Ende wohl'ger Dunkel\[C]heit,
die uns die \[Am]Sorgen die uns eng um\[Dm]flechten, 
vergessen \[E]lässt und sonderbar be\[(Am)]freit. \rrep
\endchorus

\beginverse
Ein letzter ^Blick verbogene Kron^korken,
ein langes ^Haar, das kann von mir nicht ^sein.
Wie geht es dir wohl nun am ander'n ^Morgen?
Auf meinen ^Tisch strahlt sanft der Sonnenschein.
\endverse

\printchorus

\endsong
