\beginsong{Griechischer Wein}[mel={Udo Jürgens}, txt={Michael Kunze}, jahr=1974, index={Es war schon dunkel}]

\beginverse
Es war schon \[Em]dunkel, als ich durch Vorstadtstraßen \[C]heim\[D]wärts \[G]ging.
Da war ein Wirtshaus, aus dem das Licht noch auf den \[G]Geh\[C]steig \[D]schien.
Ich hatte \[Em]Zeit und mir war \[H]kalt, drum trat ich \[Em]ein.

Da saßen \[Em]Männer mit braunen Augen und mit \[C]schwar\[D]zem \[G]Haar,
und aus der Jukebox erklang Musik, die fremd und \[G]süd\[C]lich \[D]war.
Als man m\[Em]ich sah, stand \[H]einer auf und lud mich \[Em]ein.
\endverse

\beginchorus
\[C]Griechischer Wein ist so wie das Blut der Erde.
\[G]Komm', schenk dir ein, und wenn ich dann traurig werde,
\[D]liegt es daran, dass ich immer träume von \[G]daheim... Du musst verzeih'n.

\[C]Griechischer Wein, und die altvertrauten Lieder.
\[G]Schenk' noch mal ein, denn ich fühl' die Sehnsucht wieder,
\[D]in dieser Stadt werd' ich immer nur ein Fremder \[Em]sein\[H], und al\[Em]lein.
\endchorus

\beginverse
Und dann ^erzählten sie mir von grünen Hügeln, ^Meer ^und ^Wind,
von alten Häusern und jungen Frauen, die al^lei^ne ^sind,
und von dem ^Kind, das seinen ^Vater noch nie ^sah.

Sie sagten ^sich immer wieder: Irgendwann geht ^es ^zu^rück.
Und das Ersparte genügt zu Hause für ein ^klei^nes ^Glück.
Und bald denkt ^keiner mehr da^ran, wie es hier ^war.
\endverse

% \renewcommand{\everychorus}{\textnote{\bf Refrain (wdh.)}}
% \beginchorus
% \endchorus
% \renewcommand{\everychorus}{\textnote{\bf Refrain}}

\endsong
