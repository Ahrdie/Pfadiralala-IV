\beginsong{Gute Nacht, Freunde}[wuw={Reinhard Mey, 1972}, pfii={181}]

\markboth{\songtitle}{\songtitle}

\beginverse
\endverse

\centering\includegraphics[width=1\textwidth]{Noten/Lied048a.pdf}	

\beginverse
Habt Dank für die Zeit, die ich mit euch verplaudert hab',
und für eure Geduld, wenn's mehr als eine Meinung gab,
dafür, dass ihr nie fragtet, wann ich komm' oder geh',
für die stets off'ne Tür, in der ich jetzt steh'.
\endverse

\beginchorus
Gute Nacht, Freunde, es wird Zeit für mich zu geh'n.
Was ich noch zu sagen hätte, dauert eine Zigarette und ein letztes Glas im Steh'n.
\endchorus

\beginverse
Für die Freiheit, die als steter Gast bei euch wohnt,
habt Dank, dass ihr nie fragt, was es bringt und was es lohnt.
Vielleicht liegt es daran, dass man von draußen meint,
dass in euren Zimmern das Licht wärmer scheint.
\endverse

\endsong

\beginscripture{}
Mey schrieb das Lied unter dem Pseudonym „Alfons Yondraschek“ für das Gesangsduo Inga und Wolf, das damit im Vorentscheid zum Eurovision Song Contest 1972 den vierten Platz belegte. Es kam in den deutschen Charts auf Rang 22.
\endscripture

\begin{intersong}

\end{intersong}