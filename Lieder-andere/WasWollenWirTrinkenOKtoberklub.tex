%!TEX root = ../Single-Song.tex
\beginsong{Was wollen wir trinken}[txt={Oktoberklub, 1977}, mel={Son Ar Chistr (Originaltitel), 1929}]

\markboth{\songtitle}{\songtitle}

\beginverse
\lrep Was wollen wir \[Am]trinken, dieser Kampf war \[G]lang, 
was wollen wir \[F]trinken auf \[G]diesen \[Am]Sieg. \rrep 
\lrep Am Rotem \[C]Platz steht \[Dm]Corva\[C]lán, 
auf unsere \[Am]Sache stößt er mit uns \[G]an, 
wir trinken auf \[F]Luis \[G]Corva\[Am]lán. \rrep
\endverse

\beginverse
\lrep Dann wieder die ^Arbeit, braucht uns alle ^Mann, 
dann wieder die ^Arbeit, ^die sich ^lohnt. \rrep
\lrep Sie fordert ^Kraft, sie ^macht uns ^stark
für unsere ^Sache, dass sie weiter^geht, 
für unsere ^Soli^dari^tät. \rrep
\endverse

\beginverse
\lrep So wollen wir ^kämpfen für den nächsten ^Sieg,
so wollen wir ^kämpfen für ^unsre ^Welt. \rrep
\lrep Auf roten ^Plätzen ^singt das ^Volk 
von unserer ^Sache, die nimmt ihren ^Lauf,
die Revolu^tion hält ^keiner ^auf. \rrep
\endverse

\endsong

\beginscripture{}
Die Melodie zu diesem Lied taucht zum ersten Mal 1929 auf. Ab den 1970ern wird es immer wieder mit verschieden Texten versehen und feiert mehrere Erfolge.
\endscripture

\begin{intersong}

\end{intersong}