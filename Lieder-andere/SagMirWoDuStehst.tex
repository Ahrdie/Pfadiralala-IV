%!TEX root = ../Single-Song.tex
\beginsong{Sag mir, wo du stehst}[wuw={Oktoberklub, 1967}]

\markboth{\songtitle}{\songtitle}

\renewcommand{\everychorus}{\textnote{\bf Refrain (2x)}}
\beginchorus
\[Em]Sag mir, wo du \[D]stehst, \[Em]sag mir, wo du \[D]stehst,
\[Em]sag mir, wo du \[D]stehst, und welchen \[C]Weg du \[H7]gehst
\endchorus

\beginverse\memorize
Zu\[Em]rück oder \[G]vorwärts; du \[D]musst dich ent\[G]schließen,
wir \[H7]bringen die \[Em]Zeit nach \[F#]vorn - Stück um \[H7]Stück.
Du \[Em]kannst nicht bei \[G]uns und bei \[D]ihnen ge\[G]nießen,
denn \[H7]wenn du im \[Em]Kreis gehst, dann \[Am]bleibst du \[H7]zu\[Em]rück.
\endverse


\beginverse
Du ^gibst, wenn du ^redest, viel^leicht dir die ^Blöße,
noch ^nie über^legt zu ^haben wo^hin.
Du ^schmälerst durch ^Schweigen die ^eigene ^Größe.
Ich ^sag Dir: Dann ^fehlt deinem ^Le^ben der ^Sinn.
\endverse

\beginverse
Wir ^haben ein ^Recht darauf ^dich zu er^kennnen,
auch ^nickende ^Masken ^nützen uns ^nichts.
Ich ^will beim ^richtigen ^Namen dich ^nennen,
und ^darum ^zeig mir dein ^wah^res Ge^sicht.
\endverse


\endsong

\beginscripture{}
\endscripture

\begin{intersong}

\end{intersong}