\beginsong{Titel des Liedes}[
	index={Alternativer Titel nur für die sortierung},
	wuw={Worte und Weise(alternativ: ww)<natürlich nur, wenn beides von der gleichen Person stammt.>},
	jahr={jahr(alternativ: j, jahr)},
	mel={Autor der Melodie(mel, melodie, weise)},
	meljahr={jahr(meljahr, melj, weisej, weisejahr)},
	txt={Autor des Textes(txt, text, worte)},
	txtjahr={Jahr(txtj, textj, txtjahr, textjahr, wortej, wortejahr)},
	alb={Das Album (alb, album)},
	lager={Lager auf dem Das Lied geschrieben wurde},
	tonart={Tonart des Liedes (tonart, key)},
	bo={Seite im Liederbock(bo, bock, Liederbock)},
	siru={SEite in der Singenden Runde},
	biest={SEite im Biest},
	gruen={Seite im Grünen},
	cp={Seite im Codex Pathomomosis (oder wie auch immer der sich schreibt)},
	pfi={Seite Im Pfadiralala1 (PF1, PF, PFI)},
	pfii={seite im Pfadiralala2 (PF2, PFII)},
	pfiii={analog (PF3, PFIII)},
	pfiv={analog (PF4, PFIV, PFIIII)},
	pfivp={Seite Im Pfadiralala4 plus (PF4P, PFIVP, PFIIIIP)},
	ju={SEite in der Jurtenburg (JU, JURTEN, JURTENBURG)},
	kssiv={SEite in Kinder SChoko songs 4 (KSS4, KSSIIII, KSSIV)},
	eg={Seite im evangelischen Gesangbuch (EG, EVG)},
	egplus={SEite im evangelischen GEsangbuch plus (EGP, EVGP, EGPLUS, EVGPLUS)},
	tf={Seite im Turmfalken (TF, TURM, TURMFALKE)},
	gb={Site im Gnorkenbüdel (GB, GNORKEN, GNORKENBÜDEL)},
]


\beginverse
D\[Am]as hier\[Em] ist d\[F]ie erste Str\[C]ophe,
Man\[Esus4] schreibt \[H]die Akko\[G7]rde einfach
ü\[E]ber de\[Dm]n Text.\[H] Hier in Großkleinschreibung
\endverse

\beginverse*
Das \[G]hier \[F]ist au\[Em]ch ein\[Dm]e Stro\[Cm]phe,
a\[G]lerdings o\[H]hne N\[Cm]ummer und mit m-Notation
\endverse

\beginchorus
Da\[Esus2]s ist der Refrain. Der funktioniert wie alle Strophen auch.
beachte: {\dq}Akk\[A]orde{\dq} werden linksb\[B]ündig eingefügt, sowohl im Refrain, als auch in den Strophen.
\endchorus

\beginverse
Die Nummerierten strophen werden weitergezählt.
Das hier ist also die zweite Strophe. Hier stehen
keine Akkorde.
N\[C]ur in d\[D]er le\[E]tzten zeile stehen Akkorde
\endverse

\beginverse
 \lrep  D\[C#]ie Dritte S\[E5]trophe hat wieder Akk\[C]orde \[D]\hspace{0.67em}\[Dsus2]\hspace{2.00em}\[D]\hspace{0.67em}\[Em] 
war\[C]um auch immer. Vielleicht ist die Melodie \[Am]anders  \rrep 
\endverse

\beginverse*
dieser Vers hat keine Akkorde. Beachte, dass durch den leerraum eine 
ne\[D]ue Strophe \[E]ohne Nummer entsteht. D\[F#m]ieser Vers hat Akkorde
so schreibt man einen Vers ohne Akkorde.
\endverse

\endsong

\beginscripture{}
Im Prinzip kann man den Infoblock überall hinsetzen.
Der infoblock enthält zusätzlich Informationen über das lied, zum 
Beispiel zur Geschichte.
Am sinnvollsten ist er unter dem Lied untergebracht. Der Infoblock
beginnt mit @Info: oder Info:
Auch mehrere Zeilen sind drin.
Leerzeilen sind aber nicht drin (sonst gibt es eine neue strophe)
Das heißt aber noch lange nicht, dass sich Latex auch daran hält.
Neuzeilen gehen aber mit \\
Im Prinzip sind also Latex-kommandos möglich
\endscripture
